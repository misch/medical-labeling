
\documentclass[oneside,a4paper]{book}
%\pagestyle{headings}


%\usepackage[labelfont=bf, font={sf,normalsize}, margin=0.5cm]{caption}
%\usepackage{caption}

%=============================================================================

\usepackage{amsthm}
\usepackage{xspace}
\usepackage{float}
\usepackage{ifthen}
\usepackage{amsbsy}
\usepackage{amssymb}
\usepackage{balance}
\usepackage{booktabs}

% show real images
\usepackage[pdftex]{graphicx}
%\usepackage[draft]{graphicx}

\usepackage{rotating}
\usepackage{multirow}
\usepackage{needspace}
\usepackage{microtype}
\usepackage{bold-extra}
\usepackage{geometry}
%\geometry{
%a4paper,
%total={140mm,237mm},
%}
\usepackage{varioref}
\usepackage{xcolor}
\usepackage{textcomp}
\usepackage{listings}
\usepackage[normalem]{ulem} %emphasize still italic
\usepackage{ucs}

% \usepackage[utf8]{inputenc}
% \usepackage[htt]{hyphenat}
\usepackage{times}
\usepackage{url}
\usepackage{alltt}
\usepackage{amsmath}
\usepackage{xfrac}
%\usepackage{subfigure}
\usepackage{appendix}
\usepackage{stmaryrd}   % for the \shortuparrow
\usepackage[utopia]{quotchap}

\usepackage{setspace}
\usepackage[numbers, sort&compress]{natbib}
\usepackage{mdwlist}        % support for better spaced lists
% allows for temporary adjustment of side margins
\usepackage{chngpage}
\usepackage[normalem]{ulem} 

\usepackage{subcaption}
\usepackage{amsfonts}
\usepackage{here}

% for fancy todo
\usepackage{todonotes}
\newcommand{\todoRef}{\todo[color=green!20]}
\newcommand{\todoEnglish}{\todo[color=red!20]}
\newcommand{\todoFormat}{\todo[color=blue!20]}
\newcommand{\todoWriteMore}{\todo[color=yellow!40, inline]}
\newcommand{\todoCheck}{\todo[color=green!20]{check!}}

% constants

\newcounter{qcounter}

% commands
\newcommand{\n}{$\cdot$}
\newcommand{\y}{\checkmark}
\newcommand{\subscript}[1]{$_{\textrm{\footnotesize{#1}}}$}
\newcommand{\superscript}[1]{$^{\textrm{\footnotesize{#1}}}$}
\newcommand{\vertical}[1]{\raisebox{-4em}{\begin{sideways}{#1}\end{sideways}}}

\newboolean{showedits}
\setboolean{showedits}{true} % toggle to show or hide edits
\ifthenelse{\boolean{showedits}}
{
       \newcommand{\ugh}[1]{\textcolor{red}{\uwave{#1}}} % please rephrase
       \newcommand{\ins}[1]{\textcolor{blue}{\uline{#1}}} % please insert
       \newcommand{\del}[1]{\textcolor{red}{\sout{#1}}} % please delete
       \newcommand{\chg}[2]{\textcolor{red}{\sout{#1}}{\ra}\textcolor{blue}{\uline{#2}}} % please change
}{
       \newcommand{\ugh}[1]{#1} % please rephrase
       \newcommand{\ins}[1]{#1} % please insert
       \newcommand{\del}[1]{} % please delete
       \newcommand{\chg}[2]{#2}
}


% ============================================================================
% Put edit comments in a really ugly standout display

\usepackage{xcolor}
\usepackage[normalem]{ulem}
\newcommand{\ra}{$\rightarrow$}


% comments \nb{label}{color}{text}
\newboolean{showcomments}
\setboolean{showcomments}{true}
\ifthenelse{\boolean{showcomments}}
    {\newcommand{\nb}[3]{
        {\colorbox{#2}{\bfseries\sffamily\scriptsize\textcolor{white}{#1}}}
        {\textcolor{#2}{\sf\small$\blacktriangleright$\textit{#3}$\blacktriangleleft$}}}
     \newcommand{\version}{\emph{\scriptsize$-$Id$-$}}
%	 \newcommand{\ugh}[1]{\textcolor{red}{\uwave{#1}}} % please rephrase
%	 \newcommand{\ins}[1]{\textcolor{blue}{\uline{#1}}} % please insert
%	 \newcommand{\del}[1]{\textcolor{red}{\sout{#1}}} % please delete
%	 \newcommand{\chg}[2]{\textcolor{red}{\sout{#1}}{\ra}\textcolor{blue}{\uline{#2}}} % please change
	 \newcommand{\chk}[1]{\textcolor{ForestGreen}{#1}} % changed, please check
	}
    {\newcommand{\nb}[3]{}
     \newcommand{\version}{}
	\newcommand{\chk}[1]{} % changed, please check
	}

% ============================================================================
% Make quotes be italic
\renewenvironment{quote}
    {\list{}{\rightmargin\leftmargin}%
     \item\relax\begin{it}}
    {\end{it}\endlist}

\newcommand{\ttimes}{\ensuremath{\times}}

%=============================================================================

\newcommand{\needlines}[1]{\Needspace{#1\baselineskip}}

% source code
\usepackage{xcolor}
\usepackage{textcomp}
\usepackage{listings}
\definecolor{source}{gray}{0.9}
\lstset{
	language={},
	% characters
	tabsize=3,
	upquote=true,
	escapechar={!},
	keepspaces=true,
	breaklines=false,
	alsoletter={:},
	breakautoindent=true,
	columns=fullflexible,
	showstringspaces=false,
	basicstyle=\footnotesize\ttfamily,
	% background
	frame=single,
    framerule=0pt,
	backgroundcolor=\color{source},
	% numbering
	numbersep=5pt,
	numberstyle=\tiny,
	numberfirstline=true,
	% captioning
	captionpos=b,
	numberbychapter=false,
	% formatting (html)
	moredelim=[is][\textbf]{<b>}{</b>},
	moredelim=[is][\textit]{<i>}{</i>},
	moredelim=[is][\uline]{<u>}{</u>}}
\newcommand{\ct}{\lstinline[backgroundcolor=\color{white},basicstyle=\footnotesize\ttfamily]}
\newcommand{\lct}[1]{{\small\tt #1}}


%----------------------------------------------------------------------------
% references
\newcommand{\tabref}[1]{\hyperref[{tab:#1}]{Table~\ref*{tab:#1}}}
\newcommand{\figref}[1]{\hyperref[{fig:#1}]{Figure~\ref*{fig:#1}}}
\newcommand{\secref}[1]{\hyperref[{sec:#1}]{Section~\ref*{sec:#1}}}
\newcommand{\lstref}[1]{\hyperref[{lst:#1}]{Listing~\ref*{lst:#1}}}
\newcommand{\charef}[1]{\hyperref[{cha:#1}]{Chapter~\ref*{cha:#1}}}
%----------------------------------------------------------------------------

% abbreviations
\tracingcolors 4
\setcounter{tocdepth}{3}
\setcounter{secnumdepth}{3}
\newcommand{\ie}{\emph{i.e.,}\xspace}
\newcommand{\eg}{\emph{e.g.,}\xspace}
\newcommand{\etc}{\emph{etc.}\xspace}
\newcommand{\etal}{\emph{et al.}\xspace}


\newcommand{\newevenside}{
	\ifthenelse{\isodd{\thepage}}{\newpage}{
	\newpage
        \phantom{placeholder} % doesn't appear on page
	\thispagestyle{empty} % if want no header/footer
	\newpage
	}
}

\def\stretchfactor{1}
\newcommand{\mychapter}[1]{\setstretch{1}
    \chapter{#1}\setstretch{\stretchfactor}}

%----------------------------------------------------------------------------
\newcommand{\lessSpace}{\vspace{-1em}}
\DeclareGraphicsExtensions{.pdf,.png}
\graphicspath{{Figures/}}
\newcommand{\fig}[4]{
	\begin{figure}[#1]
		\centering
		\includegraphics[width=#2\textwidth]{#3}
		\lessSpace
		\caption{\label{fig:#3}#4}
	\end{figure}}

% ===========================================================================


\newcommand{\thesistitle}{Towards passive labeling in 3D via gaze observation}
\newcommand{\thesisauthor}{Mich\`ele Wyss}
\newcommand{\thesisleiter}{Prof. Dr. Paolo Favaro}
\newcommand{\thesisasst}{Dr. Raphael Sznitman}
\newcommand{\thesissubtitle}{gathering annotations in medical images}
\newcommand{\thesisdate}{April 2016}



% ===========================================================================

\usepackage[ colorlinks=true, urlcolor=black, linkcolor=black,
			citecolor=black, bookmarksnumbered=true, bookmarks=true,
			plainpages=false,
			pdftitle={\thesistitle}, pdfauthor={\thesisauthor},
			pdfsubject={\thesissubtitle}, pdfpagelabels]{hyperref}

\newcommand{\hrref}[2]{\hyperref}
% ===========================================================================
% ===========================================================================


% D O C U M E N T
% % % % % % % % % % % % % % % % % % % % % % % % % % % % % % % % % %
\begin{document}

% T I T L E
% % % % % % % % % % % % % % % % % % % % % % % % % % % % % % % % % %
\begin{titlepage}  
  \begin{center}  
  
  \begin{figure}[t]  
  \vspace*{-1cm}        % to move header logo at the top 
  \center{\includegraphics[scale=0.2]{logos/MSc_quer.png}}
  \vspace{0.4in}     
  \end{figure}

    \thispagestyle{empty}
    
    {\bfseries\Huge \thesistitle \par
    \Large \vspace{0.1in} \thesissubtitle \par}

    \vspace{0.3in} 
    \LARGE{\textbf{Master Thesis} \\}
    \vspace{0.4in}

    {\Large \thesisauthor}
    
    \vspace{0.3in}
    %{\Large Home University \par}
    {\Large Philosophisch-naturwissenschaftliche Fakult\"{a}t \\
            der Universit\"{a}t Bern \par}
    \vfill
    {\Large \thesisdate \par}
  

  \vspace{0.9in}
 
  % === Logos ==============================================     
  \begin{figure}[htp]
    \centering
    \includegraphics[scale=0.30]{logos/UNI_Bern.png}\hfill
    \includegraphics[scale=0.30]{logos/UNI_Neuenburg.png}\hfill
    \includegraphics[scale=0.80]{logos/UNI_Fribourg.png}
  \end{figure}
  % === // Logos ===========================================    


  \end{center}

\end{titlepage}


% A B S T R A C T
% % % % % % % % % % % % % % % % % % % % % % % % % % % % % % % % % %
\chapter*{\centering Abstract}
\begin{quotation}
\noindent 
Clinical decisions are nowadays often based on medical imaging and the desire for automated systems to support clinicians is high. 
Yet, using supervised learning techniques requires a vast amount of pre-annotated training data.
Producing annotations in medical images is a time-consuming task that requires experienced experts and is a critical bottleneck for using Machine Learning techniques in medical image analysis. 
We discuss a previous attempt to use eye-tracking to produce ground truth data and we show the difficulties that come with this great potential for time cost reduction. 
We overcome the limitations by reformulating the underlying problem as a PU-learning problem. The suggested loss function to solve this problem can be optimized using gradient boosting.
Our approach outperforms the other discussed approach on real and synthetic data and brings us a step towards automated segmentation using eye-tracking.


\end{quotation}
\clearpage


% C O N T E N T S 
% % % % % % % % % % % % % % % % % % % % % % % % % % % % % % % % % % % % % % % %
\section*{Acknowledgement}
I would like to thank my supervisor, Prof.\ Dr.\ Paolo Favaro of the Institute for Computer Science at Universit\"at Bern for supporting me to work on this project outside of the Institute. I would also like to thank my Co-supervisor, Dr.\ Raphael Sznitman of the ARTORG Center for Biomedical Engineering Reserach at Universit\"at Bern for the great support, encouragement and inputs during this project. Raphael was there to discuss ideas and plan further steps, and he was of great help whenever I ran into problems with the project or writing. Further, I would like to thank the other people of the Ophthalmic Technology Laboratory at the ARTORG Center who provided appreciated inputs when I was preparing my presentations or trying to figure out how to tackle certain problems. My special thanks go to Stefanos Apostolopoulus who gave me his code for collecting the eye-tracking data from the device, and who fixed the crashing server for my calculations. Also, I would like to thank my partner and my dear friends and colleagues for listening to my unclear thoughts, for keeping me motivated, for proof-reading my written thesis and for understanding that my ``only half an hours'' took sometimes longer than expected. Finally, I would like to express my profound gratitude to my parents and my brother for providing me with unfailing support and encouragement throughout my years of study and through the process of writing this thesis. Thank you.

\vspace{1cm}
\noindent Author

\vspace{1cm}
\noindent Mich\`ele Wyss

\tableofcontents


% Chapter 1
\chapter{Introduction}
\label{chap:introduction}

\section{Motivation}
      Medical imaging is a set of techniques to produce visual representations of the interior of a body. 
      Such images are used for clinical analysis and medical intervention, e.g. to diagnose and treat disease. 
      Techniques like Computed Tomography (CT) or Magnetic Resonance Imaging (MRI) reveal internal structures hidden by skin and bones. 
      Clinical decisions are nowadays often based on these data, and the desire for automated systems to support data analysis is high. 
      Therefore, creating automated systems to support clinicians in their everyday work, that is e.g.\ for diagnosis or treatment purposes, has been an important topic in biomedical engineering research for a few decades.
      However, the data situation in medicine is not satisfying: 
      Whereas we have in general large amounts of data available, gathering annotations in those data for later usage in supervised learning settings is a bottleneck. 
      Following up on an idea of Vilari\~no et al.\ \cite{vilarino2007automatic}, we apply eye-tracking to get expert knowledge about the data and therefore allow clinicians to annotate data in their everyday life, without losing a tremendous amount of time and doing the tedious and unrewarding work of annotating data manually.
  
\section{Goal}
The ultimate goal of this research project is to produce reliable ground truth annotations / segmentations for future supervised training at low time costs. 
This thesis elaborates... \todo{?}

\section{Overview}
Chapter \ref{chap:background} gives a short overview about general notions considering classification in Machine Learning, segmentation approaches in medical imaging and also, the approach and implicit assumptions of Vilari\~no et al.\ will be briefly elaborated. In Chapter \ref{chap:characterizing-gaze} it will be discussed how well observed gaze positions acquired with a rather low-cost device can be used to gain information about the positive class. Chapter \ref{chap:learning-with-unlabeled-data} will further discuss the problem that the used data acquisition method using gaze observations will lead to a limited amount of labels -- namely we are dealing with having only positive labels and no further information about negative ones which is a so-called PU-learning problem. 
Chapter \ref{chap:conclusion} quickly summarizes the key conclusions of this work and gives an outlook of what can be elaborated in future work.



% Chapter 2
\chapter{Background}
\label{chap:background}
This chapter will briefly explain the method of Vilari\~no et al. \cite{vilarino2007automatic} to use gaze-tracking for polyp detection as well as elaborate their assumptions, limitations and conceptual problems that arise when trying to generalize their methods to arbitrary data gained through medical imaging methods. Further, there will be given a brief overview and explanation of classification in general and the used gradient boosting method.

\section{Classifiers and Boosting}
In classification, the task is to find for a given data sample $x \in \mathbb{R}^n$ a class label $f(x) \in \{c_1, c_2, ..., c_k\}$. As an example, $x$ could be a representation of an e-mail and $\{c_1,c_2,c_3\}$ could be \{``spam'', ``business'', ``private''\}. For simplicity, usually a binary classifier with $f(x) \in \{-1,1\}$ is considered and multiclass problems are later formulated in terms of the two-class approach. 
Given the true label $y$ for a sample $x$ and the prediction $f(x)$, the concept of a ``margin'' describes the confidence of the made prediction $f(x).$ 
It is foften defined as $z = y f(x)$.
Note that, as described until now, the margin would be
\begin{equation*}
z = 
     \begin{cases}
	+1, \quad \text{if } y = f(x), \\
	-1 \quad \text{if } y \neq f(x),
      \end{cases}
\end{equation*}
indicating that a negative value of the margin means that the prediction $f(x)$ was wrong. Usually, instead of $f: \mathbb{R}^n \longrightarrow \{-1,1\}$ we have a function $g$ predicting ``scores'' of a sample $x$, and depending on those scores, the function $f$ decides the label (e.g.\ $f(x) = \text{sign}(g(x))$). The margin is then calculated for the function $g$, the meaning staying the same: A large margin represents a confident decision, whereas a margin close to $0$ means a very insecure decision. A negative margin represents a wrong decision.

Finding the function $f^*$ that optimally assigns positive and negative class labels to all possible inputs has been a core topic in Machine Learning for a long time (\cite{cover1967nearest}, \cite{cox1958regression}, \cite{russell1995modern}, \cite{cortes1995support}, \cite{freund1997decision}). 
These so-called supervised learning methods depend on the availability of labeled training data to calculate e.g.\ margins and make them as big as possible, whereas unsupervised methods aim at making sense of the data without any examples (\cite{macqueen1967some}, \cite{cheng1995mean}. 
Vilari\~no et al.\ used gaze observations to directly generate training examples for the widely used Support Vector Machine (SVM) classifier \cite{cortes1995support} to show some first promising results. 
This classifier optimizes a linear\footnote{Nonlinear decision boundaries can be obtained by using kernels which give the SVM classifier a cheap possibility to compute a linear boundary in a nonlinearly deformed space \cite{boser1992training}.} boundary between the classes of a given training set by maximizing the margins of the training samples to the decision boundary. 

Instead of directly using the gaze positions as training set for a fully supervised method like a SVM classifier, we investigated in this project the meta-problem of creating correct training data from gaze positions for later usage in supervised settings. 
We formulate this meta-problem itself as a classification-/segmentation task and put it into the context of a semi-supervised setting. 
We are going to use a standard Gradient Boosting algorithm to optimize a loss function that will be formulated in an intuitive way, taking into consideration the fact that for a big part of the given data, we do not know the labels.

The idea of boosting is that while a simple ``weak'' classifier might produce predictions that are only slightly better than random guessing, combining several weak classifiers could produce a powerful committee. 
Boosting sequentially applies the weak classifier (e.g.\ decision trees or neural nets) to modified versions of the original input data. 
The modification of the input data in each step depends on the previously generated classifier; observations that were misclassified by the previous classifier get higher weights and the weights for already correctly classified observations are decreased. More specifically, the new weights depend on the margins that were obtained in the previous iterations. 
A very popular boosting algorithm is the so-called AdaBoost algorithm introduced by Freund and Schapire in 1997 (\cite{freund1997decision}). 
From a statistical point of view, this algorithm minimizes the exponential loss of the margin
$$l(z) = \sum_i e^{-z_i}.$$
This loss function gives a very high penalty to negative margins (i.e.\ wrongly classified samples). 
It turned out that this is not always the desired thing to do; especially if the training data contain outliers, concentrating on the correct classification of those leads to a bad generalizatin error and it might be better to tolerate a wrong label for this sample and maintain instead good classification boundaries for the other samples. 
That AdaBoost can be interpreted as an optimization algorithm (more precisely, forward stagewise additive modeling) based on exponential loss was discovered only later, and with it the desire to develop simple feasible boosting algorithms for arbitrary loss functions. 
However, for arbitrary (convex, differentiable) loss functions it is not trivial to solve the optimization problem arising in each step of the forward stagewise additive modeling approach, and these other loss functions do not necessarily directly lead to elegant boosting algorithms. 
In each step, what has to be found are the parameters for the optimal weak classifier (e.g.\ a decision tree) that minimizes the loss function, if added to the current model. 
For exponential loss, this simplifies to a weighted exponential criterion for the new tree. A greedy recursive-partitioning algorithm can be used with the weighted exponential loss as a splitting criterion. 

On the other hand, finding the unconstrained minimum of an arbitrary differentiable convex function could be done via numerical optimization methods such as steepest gradient descent. 
Unfortunately, the therefore needed gradients are defined only on the training data points, whereas the main goal is to generalize from the training set to unseen samples. 
A way to get a predictive model (i.e.\ a model that generalizes to previously unseen data) is, to induce a basic classifier (decision tree) at each iteration that is fit to the calculated gradients. This ``modification'' of a steepest decent method leads to the so-called gradient boosting algorithm that allows optimizing arbitrary convex loss functions.
This explanation is mainly taken from \cite{friedman2009elements}. For more detailed explanations as well as proofs of the above mentioned properties of the AdaBoost algorithm, please refer to \cite[Chapter~10]{friedman2009elements}, \cite{mason1999boosting} and \cite{friedman2001greedy}.

\section{Build a Classifier Using Gaze-Tracking}
In 2007, Vilari\~no et al. published a method that used gaze-tracking for polyp detection (\cite{vilarino2007automatic}). 
The core idea of this paper was to train a classifier using expert's gaze positions to automatically generate training samples, instead of manually labeled training data. 
They chose a very simple approach to achieve this and assumed for each video frame that the gaze position indicates a true positive image region and therefore declared a $128\times128$-pixels image patch as their positive sample. All the other non-overlapping regions of the frames were declared as negative samples.
The availability of labeled training data is the bottleneck for enabling Machine Learning and Computer Vision methods for clinical applications. 
The idea of Vilari\~no et al. has a great potential to tackle this issue because it is aimed at reducing the time costs to create training data for classifiers. 
The results reported in \cite{vilarino2007automatic} were, even if not yet suitable for clinical application, promising (see performance plots in Figure \ref{fig:vilarino-results}).
\begin{figure}[ht]
	\centering
	\includegraphics[width=\textwidth]{vilarino_results_ROC-PR}
	\caption{The reported results in \cite{vilarino2007automatic} are very promising, even though the authors say that it is not yet suitable for clinical applications. AUC = 0.93.}.
	\label{fig:vilarino-results}
\end{figure}

However, the application was limited to polyp detection in colonoscopy videos and especially, their way to build the training set from gaze observations implicitely used the following assumptions:
\begin{enumerate}
 \item In each video frame, there was at most one connected structure of interest (polyp).
 \item A structure of interest never exceeded the size of $128 \times 128$ pixels.
\end{enumerate}
In their case, this might be justified (examples of how their data looked like are shown in Figure \ref{fig:vilarinoPolypExamples}). 
In fact, their evaluation is not meaningful as soon as one of the above assumptions is not fulfilled. What is evaluated is simply a measure of how well a SVM classifier can separate observed from non-observed image patches. 

\begin{figure}[ht]
	\centering
	\includegraphics[width=\textwidth]{vilarino-polyp-examples}
	\caption{Examples of image patches containing polyps in Vilari\~no et al.'s work. The pictures are taken from \cite{vilarino2007automatic}.}
	\label{fig:vilarinoPolypExamples}
\end{figure}
Using datasets that, more or less, fulfill assumptions 1 and 2 from above, we achieved visually reasonable results regarding detection with their suggested approach (see Figure \ref{fig:theirapproachairplane}). \todo{add better description of figure!}
As we are aiming for a pixel-/voxelwise segmentation of the input rather than a classification of whole image patches, we have to capture the object boundaries and we therefore decided to perform a pre-segmentation using SLIC superpixels (\cite{achanta2010slic}) as implemented in the VLFeat library \cite{vedaldi08vlfeat}. Some example outputs are shown in Figure \ref{fig:airplaneSLIC}. 

\begin{figure}[ht]
	\centering
	\begin{subfigure}[h]{0.31\textwidth}
		\includegraphics[width=\textwidth]{airplane-input-frame_00189}	
		\caption*{input (frame 189)}
	\end{subfigure}
	~
	\begin{subfigure}[h]{0.31\textwidth}
		\includegraphics[width=\textwidth]{airplane-binaryOutput-frame189-svm-patches-c10}	
		\caption*{binary output}
	\end{subfigure}
	~
	\begin{subfigure}[h]{0.31\textwidth}
		\includegraphics[width=\textwidth]{airplane-heatmapOutput-frame189-svm-patches-c10}	
		\caption*{heat map}
	\end{subfigure}
	
	\vspace{3mm}
	\begin{subfigure}[h]{0.31\textwidth}
		\includegraphics[width=\textwidth]{airplane-input-frame_00249}	
		\caption*{input (frame 249)}
	\end{subfigure}
	~
	\begin{subfigure}[h]{0.31\textwidth}
		\includegraphics[width=\textwidth]{airplane-binaryOutput-frame249-svm-patches-c10}	
		\caption*{binary output}
	\end{subfigure}	
	~
	\begin{subfigure}[h]{0.31\textwidth}
		\includegraphics[width=\textwidth]{airplane-heatmapOutput-frame249-svm-patches-c10}	
		\caption*{heat map}
	\end{subfigure}	
	\caption{Inputs and outputs obtained with Vilari\~no's approach. For the shown examples, the SVM classifier of the {\tt libsvm} package for MATLAB \cite{libsvm} was used with a RBF kernel ($\gamma = 0.625$) and a rather high regularization value of $c = 10$. Note that the region containing the airplane was at least partly detected.}
	\label{fig:theirapproachairplane}
\end{figure}

\begin{figure}[ht]
	\centering
	\begin{subfigure}[h]{0.31\textwidth}
		\includegraphics[width=\textwidth]{airplane-input-frame_00189}
		%\missingfigure[figwidth=\textwidth]{airplane video input (several frames)}
		\caption*{input (frame 189)}
	\end{subfigure}
	~
	\begin{subfigure}[h]{0.31\textwidth}
		\includegraphics[width=\textwidth]{airplane-binaryOutput-frame189-svm-superpixelsColor-c10}	
		%\missingfigure[figwidth=\textwidth]{results (patch vs. superpixels}
		\caption*{binary output}
	\end{subfigure}
	~
	\begin{subfigure}[h]{0.31\textwidth}
		\includegraphics[width=\textwidth]{airplane-heatmapOutput-frame189-svm-superpixelsColor-c10}	
		%\missingfigure[figwidth=\textwidth]{results (patch vs. superpixels}
		\caption*{heat map}
	\end{subfigure}
	
	\vspace{3mm}
	\begin{subfigure}[h]{0.31\textwidth}
		\includegraphics[width=\textwidth]{airplane-input-frame_00249}	
		%\missingfigure[figwidth=\textwidth]{airplane video input (several frames)}
		\caption*{input (frame 249)}
	\end{subfigure}
	~
	\begin{subfigure}[h]{0.31\textwidth}
		\includegraphics[width=\textwidth]{airplane-binaryOutput-frame189-svm-superpixelsColor-c10}	
		%\missingfigure[figwidth=\textwidth]{results (patch vs. superpixels}
		\caption*{binary output}
	\end{subfigure}	
	~
	\begin{subfigure}[h]{0.31\textwidth}
		\includegraphics[width=\textwidth]{airplane-heatmapOutput-frame189-svm-superpixelsColor-c10}	
		%\missingfigure[figwidth=\textwidth]{results (patch vs. superpixels}
		\caption*{heat map}
	\end{subfigure}	
	\caption{Inputs and outputs obtained with a pre-segemented image using SLIC superpixels. We used color-based features as described in Chapter \ref{chap:chap:learning-with-unlabeled-data} and the same classifier as in Figure \ref{fig:theirapproachairplane}.}
	\label{fig:airplaneSLIC}
\end{figure}

However, there are countless cases where the assumptions are not fulfilled and these datasets cause conceptual and practical problems. 
In the ideal case the gaze observations give us one true positive location for each frame. 
However, considering all the other non-overlapping patches / superpixels as negative, is not valid, if one of the above-mentioned assumptions is not fulfilled.
Figure \ref{fig:nonValidAssumptionD2} shows an example where assumption 2 fails. The object to be segmented has a large extent and it is problematic to use the gaze observations in the way it was done in \cite{vilarino2007automatic}. Generating positive and negative ground truth labels by considering all regions except the ones described by the gaze leads to many positive patches / superpixels in the negative training set.

\begin{figure}[ht]
	\centering
	\begin{subfigure}[h]{0.48\textwidth}
		\includegraphics[width=\textwidth]{dataset2gazePositionFrame207}
		\caption*{frame 207 (red: gaze position)}
	\end{subfigure}
	~
	\begin{subfigure}[h]{0.48\textwidth}
	    \includegraphics[width=\textwidth]{dataset2SLICsegmentationFrame207}
	    \caption*{SLIC superpixels (red: gaze position)}
	\end{subfigure}
	
	\vspace{3mm}
	\begin{subfigure}[h]{0.31\textwidth}
		\includegraphics[width=\textwidth]{dataset2positivePatchFrame207}	
		\caption*{positive patch}
	\end{subfigure}
	~
	\begin{subfigure}[h]{0.31\textwidth}
		\includegraphics[width=\textwidth]{dataset2positiveSuperpixelFrame207}	
		\caption*{positive superpixel}
	\end{subfigure}
	
	\vspace{3mm}
		\begin{subfigure}[h]{0.48\textwidth}
		\includegraphics[width=\textwidth]{dataset2negativePatchesFrame207}	
		\caption*{negative patches}
	\end{subfigure}
	~
	\begin{subfigure}[h]{0.48\textwidth}
		\includegraphics[width=\textwidth]{dataset2negativeSuperpixelsFrame207}	
		\caption*{negative superpixels}
	\end{subfigure}	
	\caption{An example of a dataset where the contained object has a larger extent than what superpixels / $128 \times 128$-image patches could describe with one observed gaze location. Assuming that all regions except the one described by the gaze are negative leads to a noisy negative training set.}
	\label{fig:nonValidAssumptionD2}
\end{figure}

Examples of assumption 1 failing are our used datasets ``eye tumor'' and ``cochlea.'' Even though the structures are not big in extent, there are multiple interesting parts in one frame -- in the shown images, all the parts lie (mainly) within the $(128\times128)$-patch and therefore considering all the others patches as negative is not problematic. 
However, it is not clear that this stays like this during the whole sequence.
In the cochlea example it would be enough for the observer to focus on the lower left part to include positive parts in the negative set. 

\begin{figure}[ht]
	\centering
	\begin{subfigure}[h]{0.48\textwidth}
		\includegraphics[width=\textwidth]{dataset7positivePatchFrame47}
		\caption*{eye tumor: frame 47 (red: gaze position with surrounding ($128\times128$)-patch}
	\end{subfigure}
	~
	\begin{subfigure}[h]{0.48\textwidth}
	    \includegraphics[width=\textwidth]{dataset7positivePatchFrame47gt}
	    \caption*{ground truth (white: positive) \newline}
	\end{subfigure}
	
	\vspace{3mm}
	\begin{subfigure}[h]{0.48\textwidth}
		\includegraphics[width=\textwidth]{dataset8positivePatchFrame189}	
		\caption*{cochlea: frame 189}
	\end{subfigure}
	~
	\begin{subfigure}[h]{0.48\textwidth}
		\includegraphics[width=\textwidth]{dataset8positivePatchFrame189gt}	
		\caption*{ground truth}
	\end{subfigure}
	\caption{The structures of interest have more than one part -- whereas they happen to be (almost) contained withing the patch, this might not stay like this during the whole sequence. If the positive parts (white in the right column) were only slightly further away from each other, or if the observer's focus were a bit different, positive data would be included in the negative training set.}
	\label{fig:nonValidAssumptionD78}
\end{figure}

The basic problem is therefore that we can observe only one position at a time, and if there happen to be other positive parts in this very frame, we have no way so far to consider this. 
It could be argued that whatever part is not focused in one frame might be focused in a later one. 
Yet, in this case we might already have included this part in the negative set in the previous frames, possibly multiple times. 
This means that the negative training set is actually a mixture of positives and negatives.
A natural step is therefore to formulate the problem in a different way, namely not as ``separate observed (positive) from unobserved (negative) samples'', but instead as ``given some positive samples, figure out whether or not the other, unlabeled samples, belong to the positive or to the negative class.'' This is the so-called PU-learning problem (see e.g.\ \cite{elkan2008learning})

\subsection{Interactive Image Segmentation}
Boykov et al.\ published in 2006 well-known work \cite{boykov2006graph} about interactive image segmentation which comes fairly close to the problem we have to solve. 
The approach needs as input an image and user strokes indicating back- and foreground regions. 
A segmentation is achieved by optimizing a function based on estimated conditional probabilities from the user-provided seeds. 
The quality of the obtained results is very convincing and even 3D objects can be segmented very well in medical images from only a few seeds in one depth slice (see e.g.\ Figure \ref{fig:boykovbones}). 
Yet, there are examples where correcting seeds have to be added in later frames (see \cite[Section~Experimental Results]{boykov2006graph}). 
Using gaze observations we try to obtain piecewise continuous user strokes in three dimensions that would make such corrections unnecessary. However, gaining ``background'' user strokes is not straight forward in this setup and we therefore aimed towards a solution that does not need any negative strokes but instead uses the fact that the positive stroke is available over time / depth.

\begin{figure}[ht]
	\centering
	\includegraphics[width=\textwidth]{boykovbones}
	\caption{A 3D segmentation result obtained by Boykov et al. in \cite{boykov2006graph} from only the shown user strokes from one frame.}.
	\label{fig:boykovbones}
\end{figure}

\subsection{PU-Learning Problem}
The PU-learning problem recently got attention and applications in document classification (\cite{li2003learning}), time series classification (\cite{nguyen2011positive}) and bioinformatics (\cite{elkan2008learning}, \cite{yang2012positive}, \cite{yang2014ensemble}, \cite{yousef2015novel}). 
Most of these methods proceed in two steps: 
\begin{enumerate}
 \item Find reliable negative samples from the unlabeled set.
 \item Use a standard classifier to separate positives and negatives.
\end{enumerate}

This corresponds to the steps that are iteratively done in previously proposed semi-supervised setups for boosting methods like ASSEMBLE \cite{bennett2002exploiting} or SemiBoost \cite{mallapragada2009semiboost}. In both approaches, the authors iteratively assign labels to the unlabeled data using the original labeled data and labels from previous iterations, if available. They start with the assumptions that there are labeled and unlabeled data points of both classes. Therefor, given only positive labels from the beginning, we still face the problem of finding representative negative labels (step 1 above) for the initialization. Our approach is in that sense different that we will not only include reliable negative samples from the unlabeled set, but instead we include also samples about whose label we are uncertain by including a notion of certainty in the optimization problem. We designed an according loss function that can be used in a standard gradient boosting framework.

Du Plessis et al.\ \cite{plessis2014PUanalysis} showed that solving the PU-problem corresponds to minimizing a non-convex loss function, and recently they presented a convex formulation by using different loss functions for the positive and the unlabeled data samples \cite{plessis2015convex}. The PU-learning problem can also be considered a supervised learning problem with label noise, and within this context, Ghosh et al. \cite{ghosh2015making} found the same sufficient symmetry conditions for the loss function as Du Plessis et al.\ \cite{plessis2014PUanalysis} to achieve an unbiased solution to the PU-learning problem. The estimator used by Du Plessis et al. was also previously proposed by Natarajan et al.\ \cite{natarajan2013learning} in a more general work about learning with noise in negative and positive training labels, where the PU-learning problem is considered a special case of having only noise in the negative set.

Considering this interpretation of the PU-learning problem as learning with noise, it is worth to mention that AdaBoost is very sensitive to noise (see e.g.\ the discussion about the exponential loss in \cite{friedman2009elements}). Several adjustments to handle noise have been suggested: MadaBoost \cite{domingo2000madaboost} is a boosting variant that gives an upper bound to the weights in the iterative updates. This helps avoiding that noisy labels are over-emphasized. 
After some earlier suggestions, Freund found in 2009 a boosting algorithm \cite{freund2009more} that pushes the classification margin to be large but ignores samples that are hard to classify. A thorough overview about these and more suggested noise tolerant boosting algorithms have been discussed by Zhou \cite{zhou2012ensemble}. \todo{add the DH loss function}
%Gradient boosting gives us a general way to minimize a loss function, and therefore we could easily also try their suggested double hinge loss $l(z) = \max(-z,\max(0,0.5 - 0.5 z))$ (for unlabeled samples) and its composite $\tilde l(z) = l(z) - l(-z) = -z$ (for positive samples). 
%
% Chapter 3
\chapter{Characterizing gaze}
\label{chap:characterizing-gaze}
This chapter will discuss the used setup and the quality of extracted positive labels gained from the observed gaze positions. The important question to answer is, how reliable our data is with respect to the assumption that gaze observation can naturally provide us with positive labels for the data. 

\section{Setup}
We used an affordable eye-tracking device called ``The Eye Tribe''. According to the producer's website (\url{http://dev.theeyetribe.com/general/}), the device has an accuracy of at least 1 degree visual angle. We placed the device on a tripod below a [?]-inch screen \todo{screen size} and observed the screen from a distance of approximately 60cm.  \todoWriteMore{tell that scans were converted to video sequences}

\begin{figure}[ht]
	\centering

	\includegraphics[width=\textwidth]{theeyetribe}	
	\caption{user in front of ``The Eye Tribe'' (image taken from \url{http://dev.theeyetribe.com/general/})}
	\label{fig:theeyetribe}
\end{figure}

The video sequences were played on full screen. In this setup, the minimum accuracy of 1 degree corresponds to an on-screen error of approximately 1cm. 
Figure \ref{fig:onedegreecircle} gives an idea of how much this error might mean in terms of our used datasets.
The instrument dataset is less sensitive to this than e.g.\ the cochlea CT scan that has fine structures to be recognized.

\begin{figure}[ht]
	\centering
	\begin{subfigure}[h]{0.31\textwidth}
		\includegraphics[width=\textwidth]{one-degree-circle-cochlea-17pix-frame195_small_new}	
		\caption*{cochlea}
	\end{subfigure}
	~
	\begin{subfigure}[h]{0.31\textwidth}
		\includegraphics[width=\textwidth]{one-degree-circle-instrument-18_62pix-frame195_small_new}
		\caption*{instrument}
	\end{subfigure}
	~
	\begin{subfigure}[h]{0.31\textwidth}
		\includegraphics[width=\textwidth]{one-degree-circle-eyeMRI-12_28pix-frame46_small}	
		\caption*{eye tumor}
	\end{subfigure}
	\caption{Visual illustration of how much error a 1 degree visual angle causes in the different datasets using our described setup. Whereas most parts of the instrument are big in size, the fine structures in the images of the cochlea or the eye tumor might not be hit by the measured gaze position, even when an expert is looking exactly at them.}
	\label{fig:onedegreecircle}
\end{figure}

Vilari\~no et al. suggested to explore the analysis of gaze fixation patterns and voice labeling in future studies. 
In this project we want to focus on the other issues of their approach described in Chapter \ref{chap:background}. 
Therefore, we stick with their appoach and asked the user to press a key when an important structure / object is appearing in the video and to focus on the actual important object. 
Note that therefore, what is presented in this work, is an active application of eye-tracking with the potential to become passive in the future.

\section{Reliability of gaze observations}
Given some ground truth data (datasets ``instrument'', ``cochlea'', ``eye tumor''), we evaluated if the assumption of Vilari\~no et al.\ is valid, that only ``positive positions'' are hit by the gaze whenever the observer indicates (by pressing a key) that he sees a structure of interest. 
As the user had the task to focus on the object in the video, we expected that the majority of recorded gaze positions are located at true positives, or that at least some true positive points can be found within a 1 degree visual angle of the recorded gaze position. 
Clearly, the smaller the structures, the less likely it was that a true positive was hit. 
We tried here to separate the human from the measurement error by investigating, for each dataset, the actual distance between the gaze position and the closest true positive point. 
Values above 1 degree visual angle clearly mean human error whereas values below could mean human error or measurement error. 
To get a first idea of the actual measurement error, the gaze positions were measured and evaluated on a very simple dataset; a video of a non-moving black point on white background in the middle of the screen. 
Two measurements were taken. Figure \ref{fig:gazeMeasurementAccuracy} shows that for the task of staring at this black dot, the errors remain well below 1 degree, except in the case of blinking. 
The distance to the actual center of the image is on average between 5 and 7 pixels in this test video; this corresponds to a visual angle of approximately 0.25 to 0.4 degrees.

\begin{figure}[ht]
	\centering
	\begin{subfigure}[h]{0.41\textwidth}
	      \setlength{\fboxsep}{0pt}%
	      \setlength{\fboxrule}{0.5pt}%
	      \centering
	      \fbox{\includegraphics[width=\textwidth]{gazeMeasurementAccuracy2D.pdf}}
	\end{subfigure}
	~
	\begin{subfigure}[h]{0.48\textwidth}
		\includegraphics[width=\textwidth]{gazeMeasurementAccuracy1D.pdf}	
		%\caption{}
	\end{subfigure}
	\caption{For the task of staring at the black dot in the middle of the screen, the errors remain well below 1 degree. Blinking causes considerable outliers. The mean distance to the actual center of the image (dotted blue and green lines) is between 5 and 7 pixels in this test video; this corresponds to a visual angle of approximately 0.25 to 0.4 degrees.}
	\label{fig:gazeMeasurementAccuracy}
\end{figure}


\begin{figure}[ht]
	  \includegraphics[width=0.5\textwidth]{closestPositiveDataset2vid2.pdf}
	  \includegraphics[width=0.5\textwidth]{closestPositiveDataset2vid5.pdf}
	  \includegraphics[width=0.5\textwidth]{closestPositiveDataset2vid4.pdf}
	  \includegraphics[width=0.5\textwidth]{closestPositiveDataset2vid3.pdf}
	  \caption{instrument: As expected, many values are exactly zero which means that hitting the object is rather easy in this dataset. Where the gaze is not on the object, the typically big outlier values in the distance indicate that this is not due to a measurement error, but instead the eye was really not on the object.}
	\label{fig:distanceToClosestPositiveD2}
\end{figure}

\begin{figure}[ht]
	  \includegraphics[width=0.5\textwidth]{closestPositiveDataset7vid1.pdf}
	  \includegraphics[width=0.5\textwidth]{closestPositiveDataset7vid4.pdf}
	  \includegraphics[width=0.5\textwidth]{closestPositiveDataset7vid6.pdf}
	  \includegraphics[width=0.5\textwidth]{closestPositiveDataset7vid7.pdf}
	  \caption{eye tumor: The little amount of available values means that, in general, the structure of interest (tumor) is small in size (not available values mean that there are no positive values in the ground truth frame). Between frames 40 and 50 the tumor seems to be well visible and big enough in size to be rather reliably hit by the gaze.}
	\label{fig:distanceToClosestPositiveD7}
\end{figure}

\begin{figure}[ht]
	  \includegraphics[width=0.5\textwidth]{closestPositiveDataset8vid3.pdf}
	  \includegraphics[width=0.5\textwidth]{closestPositiveDataset8vid6.pdf}
	  \includegraphics[width=0.5\textwidth]{closestPositiveDataset8vid7.pdf}
	  \includegraphics[width=0.5\textwidth]{closestPositiveDataset8vid1.pdf}
	  \caption{cochlea: In this dataset, it is rather rare that the object of interest (cochlea) is actually hit by the measured gaze positions; only in few cases the distance is exactly zero. The high variation in distance to the object indicates that it is hard to follow the fine structures with the eyes, and probably only a small part of the error might be caused by inaccuracy in the measurements.}
	\label{fig:distanceToClosestPositiveD8}
\end{figure}


\begin{figure}[ht]
	  \includegraphics[width=0.5\textwidth]{fractionsDataset2vid2.pdf}
	  \includegraphics[width=0.5\textwidth]{fractionsDataset2vid5.pdf}
	  \includegraphics[width=0.5\textwidth]{fractionsDataset2vid4.pdf}
	  \includegraphics[width=0.5\textwidth]{fractionsDataset2vid3.pdf}
	  
	  \centering
	  \includegraphics[width=0.5\textwidth]{fraction-legend}
	  \caption{instrument: using SLIC superpixels}
	\label{fig:positiveFractionD6}
\end{figure}

\begin{figure}[ht]
	  \includegraphics[width=0.5\textwidth]{fractionsDataset7vid1.pdf}
	  \includegraphics[width=0.5\textwidth]{fractionsDataset7vid4.pdf}
	  \includegraphics[width=0.5\textwidth]{fractionsDataset7vid6.pdf}
	  \includegraphics[width=0.5\textwidth]{fractionsDataset7vid7.pdf}
	  
	  \centering
	  \includegraphics[width=0.5\textwidth]{fraction-legend}
	  \caption{eye tumor: using SLIC superpixels}
	\label{fig:positiveFractionD7}
\end{figure}

\begin{figure}[ht]
	  \includegraphics[width=0.5\textwidth]{fractionsDataset8vid3.pdf}
	  \includegraphics[width=0.5\textwidth]{fractionsDataset8vid6.pdf}
	  \includegraphics[width=0.5\textwidth]{fractionsDataset8vid7.pdf}
	  \includegraphics[width=0.5\textwidth]{fractionsDataset8vid1.pdf}
	  
	  \centering
	  \includegraphics[width=0.5\textwidth]{fraction-legend}
	  \caption{cochlea: using SLIC superpixels}
	\label{fig:positiveFractionD8}
\end{figure}

Therefore, giving a tolerance of 0.4 degrees seems to be a necessary step to account for the inaccuracy of the eye-tracking device. 
The experiments in Figures \ref{fig:distanceToClosestPositiveD2}, \ref{fig:distanceToClosestPositiveD7} and \ref{fig:distanceToClosestPositiveD8} show that, however, also giving this tolerance is no guarantee that each recorded gaze position will be indicating a true positive position. 
In fact, the instrument dataset (Figure \ref{fig:distanceToClosestPositiveD2}) shows rather promising results: 
In the four recorded gaze sequences, the majority of the values are exactly zero, indicating that the gaze hit the actual object. 
Measurement errors do not seem to influence the potential accuracy of our method here, as the cases where the gaze is not on the object are mostly clear outliers with a distance of more than 0.4 degrees visual angle. 
A similar conclusion holds for the eye tumor dataset (Figure \ref{fig:distanceToClosestPositiveD7}). 
The gaze observations show that between frames 40 and 50, when the tumor is big in size, it is well hit by the gaze. 
There are, best visible in the top left figure, some positions that are slightly off the object before frame 47. 
This might be a measurement error, but more likely, it is because the gaze has not yet settled at the correct position. 
The rather big distances to any positive pixels before and after this interval reflect the fact that the size of the structure of interest decreases rapidly in these frames. 
The recordings from the cochlea dataset shown in Figure \ref{fig:distanceToClosestPositiveD8} suggest, especially the observations shown in the two upper plots between frames 150 and 160\todo{explain WHY this is suggested by the plots}, that some positive information could be gained by considering additional nearby points of the recorded gaze position.
However, most of the measurements show that it is actually a rather rare event in this dataset, that the gaze really hits the object. 
The variation in distance to the object and the nature of this dataset (fine structures that are rapidly changing over time / depth) indicates that it is either hard to find, or then hard to follow the actual cochlea parts with the eyes, if not both. 
This can clearly be considered the most challenging dataset used for this project.
In this dataset, considering that the data within a 0.4 degrees visual angle is positive, would clearly increase the probability to assign real positive data to the actual gaze observations. 
However, the total area ot the positives is in most of the frames very small compared to the whole image (see e.g.\ Figure ...). 
Therefore we pay the high price of including a lot of noise into the positive training set, in some cases maybe more noise than actual positive data.

\section{Include a certain tolerance}
\todo{\"uberleitung fehlt... maybe consider the following as ``ways to give this tolerance''}Even when staring exactly on the object, not all of the superpixel's content is necessarily positive. 
This can e.g.\ be seen in the Figures \ref{fig:positiveFractionD7} and \ref{fig:positiveFractionD8} where the structures are sometimes so small that well-regularized SLIC-superpixels cannot capture them. 
In this case, reducing the size of the superpixels would help to overcome the issue. 
However, if the gaze drifts away from the object only by a few pixels, then the positive content of the superpixels dramatically decreases. 
This issue could be overcome by considering more of the surrounding superpixels to be positive (and therefore find the ``correct'' one but also introduce noise to the positively labeled data), i.e.\ giving a certain tolerance. 
Increasing the size of the superpixels is another possibility that will help, but it is paid with accuracy at the object boundaries and, because SLIC still aims at respecting the edges in the image, the true positive superpixels will typically first extend towards the direction that will not include the wrong gaze position, assuming that this lies on the other side of a strong image edge (see Figure \ref{fig:gazeOffSuperpixelSize}).

\begin{figure}[ht]
	\centering
	\begin{subfigure}[h]{0.31\textwidth}
	      \includegraphics[width=\textwidth]{superpixelSize1instrument}
	\end{subfigure}
	~
	\begin{subfigure}[h]{0.31\textwidth}
		\includegraphics[width=\textwidth]{superpixelSize2instrument}	
		%\caption{}
	\end{subfigure}
	~
	\begin{subfigure}[h]{0.31\textwidth}
		\includegraphics[width=\textwidth]{superpixelSize3instrument}	
		%\caption{}
	\end{subfigure}	
	
	\vspace{3mm}
	\begin{subfigure}[h]{0.31\textwidth}
	      \includegraphics[width=\textwidth]{superpixelSize1eye_20px}
	\end{subfigure}
	~
	\begin{subfigure}[h]{0.31\textwidth}
		\includegraphics[width=\textwidth]{superpixelSize2eye_30px}	
		%\caption{}
	\end{subfigure}
	~
	\begin{subfigure}[h]{0.31\textwidth}
		\includegraphics[width=\textwidth]{superpixelSize3eye_100px}	
		%\caption{}
	\end{subfigure}	
	\caption{Extending the superpixel size does not necessarily help to overcome the wrong or inaccurate gaze positions, as the superpixels naturally tend to respect edges and the gaze is in this case usually on the wrong side of the edge. Yellow: gaze position that is slightly off from the object}
	\label{fig:gazeOffSuperpixelSize}
\end{figure}
[kinda conclusion:]
Clearly, the simple assumption that all the user-indicated positive labels are true positives, is not valid. 
Even the rather simple case of the surgical instrument leads to noise in the acquired positive labels. The finer the structures of interest, the less reliably the gaze positions indicate true positive information. 
The problems could be overcome by [list different ways of given tolerances (e.g.\ gaussian prob. of being positive / a fixed radius within which the data are considered positive / superpixel size]. 
\todoWriteMore{if have time: try some things with smoothed gaze!}
\todoWriteMore{maybe add tolerance plots?}


% Chapter 4
\chapter{Learning with unlabeled data}
\label{chap:learning-with-unlabeled-data}
In this chapter, we will discuss the inherently noisy labels in the setup of Vilari\~no et al. and how to overcome the issues. 
The underlying conceptual problem to solve is the following: Assuming we have only true positive gaze positions (see Chapter \ref{chap:characterizing-gaze} for a discussion about this assumption), we can still not infer any reliable information about the negative labels. Instead, we are facing the so-called PU-Learning Problem that will be explained in the following.

\section{Problem formulation}
The problem of learning a classifier from positive and unlabeled data is aimed at assigning labels to the unlabeled dataset. 
It can be considered a semi-supervised learning setup: Instead of having a positive and a negative set of examples, we are given an incomplete set of positives and a set of unlabeled examples. 
The unlabeled data contains positive and negative examples which we aim to assign to either the positive or negative class. 
Usually, gradient boosting is used to minimize a loss function $L(y,f(x))$, where $y \in \{-1,1\}^m$ is a vector of labels and $x \in \mathbb{R}^{m\times n}$ is a matrix containing $n$-dimensional features. In our case of the PU-learning problem, however, not all the labels $y_i, i \in \{1,\dots,m\}$ are given. 
Instead we have only an incomplete set of positive labels ($+1$) and the rest is unknown. In order to handle this problem within the gradient boosting framework, we need some pseudo labels (see ???). Our pseudo-labels are based on the probability $p_i$ of an unlabeled training sample $x_i$ to be positive:
\begin{equation*}
 y_i = 
    \begin{cases}
	+1, \quad & \text{if } p_i \geq 0.5, \\
	-1, \quad & \text{if } p_i < 0.5.
      \end{cases}
\end{equation*}
We can easily convert the probability $p_i$ to a probability $\tilde p_i$ of having chosen the correct pseudo-label:
\begin{equation*}
 \tilde p_i = 
    \begin{cases}
	p_i, \quad & \text{if } p_i \geq 0.5, \\
	1-p_i, \quad & \text{if } p_i < 0.5.
      \end{cases}
\end{equation*}

The loss function is then defined as
\begin{equation*}
L(y,f(x)) = \underbrace{\sum_{\{i :~ y_i = 1\}} e^{-y_i f(x_i)}}_{P} \quad + \quad \gamma\underbrace{\sum_{\{i:~ y_i \text{unknown}\}} \left( \tilde p_i e^{-y_i f(x_i)} + (1-\tilde p_i) e^{y_i f(x_i)}\right)}_{U}, 
\end{equation*}
where $y_i$ denotes the (pseudo-)label\footnote{The notation does not distinguish between real labels and pseudo-labels. For the part of the sum with unknown labels, $y_i$ denotes the pseudo-labels.} of sample $x_i$, $\tilde p_i$ is the confidence that the pseudo-label $y_i$ is correct and $f(x_i)$ is the predicted score of the classifier.
Note that if we consider the fully supervised case, the U-term will be 0 and a standard exponential loss function will be optimized. 
For unlabeled data samples, the U-term of the loss function will, as commonly done, heavily penalize negative margins, if we are very confident about our pseudo-label being correct ($\tilde p_i \approx 1$) and it will penalize positive margins, if the pseudo-label is very unlikely to be correct ($\tilde p_i \approx 0$). 
Note that this extreme case can by definition of $y_i$ and $\tilde p_i$ not occur and is therefore just of explanatory value. 
If we do not know whether or not our label is correct or incorrect ($\tilde p_i = 0.5$), then what will be penalized are negative as well as positive margins, as this would mean a confident decision towards one direction based on a randomly chosen pseudo-label (in our case $y_i = 1$, if $\tilde p_i=0.5$). Figure \ref{fig:ourlossfunctionplot} shows the U-term of the loss function for different values of the probability $\tilde p_i$.

\begin{figure}[ht]
  \centering
  \includegraphics[width=\textwidth]{loss_function_different_p.pdf}	
  \caption{U-term of the loss function for different values of $p$. For samples whose label is most likely incorrect ($p \approx 0$), small margins mean correct decisions (i.e.\ different from the label) and they are therefore rewarded whereas large margins are penalized. In the case of $p \approx 1$, it is the opposite.}
  \label{fig:ourlossfunctionplot}
\end{figure}

The derivative of the loss function with respect to the classifier's scores is then given by 

\begin{equation*}
 \frac{\partial L(y_i,f(x_i))}{\partial f(x_i)} = 
    \begin{cases}
	-y_i e^{-y_i f(x_i)}, & \text{if $y_i = 1$}\\
	-\gamma \cdot \left(y_i \tilde p_i e^{-y_i f(x_i)} - y_i (1 - \tilde p_i) e^{y_i f(x_i)} \right), & \text{if $y_i$ unknown.}
      \end{cases}
\end{equation*}
Clearly, the key of this method is the usage of the probability $p_i$ for each sample $x_i$ to be a positive sample. 
It naturally gives us a way to tune the algorithm not to concentrate the same way on all wrongly classified samples, but instead embrace the fact that there is not always a 100\% certainty that we are working with the correct labels.

\section{Synthetic data}
We conducted some basic experiments on synthetic data. 
The total size of our synthetic training set contained 160 samples, 80 of which were positive and 80 negative. 
The positive training samples were equally generated from normal distributions $\mathcal{N}(\mu_1,\Sigma_1)$ and $\mathcal{N}(\mu_2, \Sigma_2)$ with 
$$\mu_1= \begin{bmatrix}2 \\ 3 \end{bmatrix}, \quad \Sigma_1 = \begin{bmatrix}0.7 & 0.2 \\ 0.2 & 0.5 \end{bmatrix}, \qquad \mu_2 = \begin{bmatrix}4.5 \\ 2 \end{bmatrix}, \quad \Sigma_2 = \begin{bmatrix} 0.2 & 0 \\ 0 & 0.2 \end{bmatrix}$$
and the 80 negative samples were generated from a normal distribution with parameters 
$$\mu_3 = \begin{bmatrix} 2\\1.5\end{bmatrix}, \quad \Sigma_3 = \begin{bmatrix}0.6 & 0.1\\ 0.1 & 0.7\end{bmatrix}$$
as visualized in Figure \ref{fig:synthetic_train_data}. The test set consists of 1000 samples that are identically distributed as the training set.
\begin{figure}[ht]
	\centering
	\begin{subfigure}[h]{0.45\textwidth}
	\includegraphics[width=\textwidth]{synthetic-gaussians-contour.pdf}	
	\end{subfigure}
	~
	\begin{subfigure}[h]{0.45\textwidth}
	\includegraphics[width=\textwidth]{synthetic-gaussians-surf.pdf}	
	\end{subfigure}
	\caption{The used distributions}
	\label{fig:synthetic-gaussians}
\end{figure}

As a reference, we optimized a standard exponential loss using all the labels from the training set with a gradient boosting method with decision tree stumps as weak learners and a shrinkage factor of $0.1.$ 
The other experiments were done with the same algorithm, but different assumptions about the available input labels. 
First, we simulated the case of separating ``observed'' from ``unobserved'' samples, i.e.\ we tried to separate a few positive samples from all the other samples. 
In the real setting, this corresponds to the naive approach of separating the image regions that were hit by the user's gaze from all the other regions. 
As expected, there is a considerable performance loss when using this approach (see Figure \ref{fig:synthetic_results}). 
Our second experiment shows the performance achieved with the standard exponential loss using 5 known positives and some pseudo-labels for the other samples. 
The pseudo-labels were assigned according to the probabilities of the known underlying distributions (see Figure \ref{subfig:pu_train}). 
To test the performance of our PU-loss function, we used the same 5 known positives and the probabilities for the unknown labels were given as they were used before to find the pseudo-labels. 
Finally, we followed the suggestion of Du Plessis et al.\ (\cite{plessis2015convex}) and used the double hinge loss and its composite as shortly described in chapter \ref{chap:background} with the same 5 known positives and the same way of assigning pseudo-labels.
Our loss-function outperforms the reference approach using the true training labels as well as the standard exponential loss with the ``correct'' pseudo-labels (that is, the pseudo-labels according to the underlying distributions). 
This can be explained with the fact that our loss function actually takes into account the confidence about the chosen pseudo-labels and adjusts the penalties accordingly. 
Optimizing the double hinge loss and its composite as suggested in \cite{plessis2015convex} yielded better results than separating observed from unobserved data points, but could not outperform our loss-function. 

\begin{figure}[ht]
	\centering
	\begin{subfigure}[h]{0.49\textwidth}
	\includegraphics[width=\textwidth]{synthetic_train_data.pdf}	
		\caption{training data (160 samples generated from the described distributions)\newline}
		\label{subfig:ref_train}
	\end{subfigure}
	~
	\begin{subfigure}[h]{0.49\textwidth}
	\includegraphics[width=\textwidth]{synthetic_pu_train_data.pdf}	
		\caption{input for the PU-loss: 5 known positives (blue), pseudo-labels (red = -1 / green = +1), probability weights $\tilde p_i$ that pseudo-labels are correct (circle radius)}
		\label{subfig:pu_train}
	\end{subfigure}
	\caption{\subref{subfig:ref_train}) training data for the standard approach \subref{subfig:pu_train}) known positives, pseudo-labels and weights used with the standard exploss and the pu-loss, respectively}
	\label{fig:synthetic_train_data}
\end{figure}

\begin{figure}[ht]
	\centering
	\begin{subfigure}[h]{0.49\textwidth}
	\includegraphics[width=\textwidth]{synthetic_results_roc.pdf}	
	\end{subfigure}
	~
	\begin{subfigure}[h]{0.49\textwidth}
	\includegraphics[width=\textwidth]{synthetic_results_pr.pdf}	
	\end{subfigure}
	\caption{Comparisons of results using different assumptions about available labels. The PU-loss outperforms the standard exploss, even when using the theoretically correct pseudo-labels.}
	\label{fig:synthetic_results}
\end{figure}


\section{On the real data}
\subsection{Labels inferred from ground truth}
Using ground truth data for the three datasets ``instrument'', ``eye tumor'' and ``cochlea,'' we checked how well the different approaches with different assumptions about available labels perform. 
Assuming that we have for a certain amount of frames / depth slices the correct superpixel labels \footnote{$y_i = 1$, if more than 50\% of the pixels contained in a superpixel are positive in the ground truth}, we again optimized a standard exponential loss with a gradient boosting algorithm using decision tree stumps as weak learners. 
This serves as a reference for our experiments.
We evaluate how well the classifier generalizes to the remaining frames on a pixelwise basis; that is, we classify whole superpixels, but at the end we are interested in the pixelwise segmentation result and compute Receiver Operator Characteristics (ROC) and Precision-Recall values on a pixelwise basis (see Figure \ref{fig:reference-known-labels}). 
This strategy allows us to compare the results to the real pixelwise ground truth instead of a ``constructed'' ground truth of e.g.\ declaring a superpixel as positive, if it contains more than 30\% or 50\% positive ground truth pixels.
Especially if superpixels contain positive and negative ground truth regions, these scores might look a bit odd due to ambiguities in sorting of the resulting scores (all pixels of one superpixel get the same score), but they still give a good idea about the actual performance of the different methods. 


%\begin{figure}[ht]
%	\centering
%	\begin{subfigure}[h]{0.45\textwidth}
%	\includegraphics[width=\textwidth]{reference_ROC.pdf}	
%		\caption*{ROC curves}
%	\end{subfigure}
%	~
%	\begin{subfigure}[h]{0.45\textwidth}
%	\includegraphics[width=\textwidth]{reference_PR.pdf}	
%		\caption*{PR curves}
%	\end{subfigure}
%	\caption{Reference curves. Blue: instrument; red: eye tumor; yellow: cochlea. Note that the precision-recall curves for indicate that the obtained classifier is most confident about positive decisions that are actually wrong. This might indicate that our features are not very suitable for the task, and it is also a hint to the fact that very similar structures appear in the images that are actually sometimes positive and sometimes negative. A further reason is that only a few frames of those datasets contain positive information at all. As the user presses the key only if he sees ``interesting'' (positive) information, we gathered the positive and negative labels from the frames with a pressed key. Therefore, the positive information might be over-represented whereas at the same time some significant negative examples (from frames that do not contain any positive information) might not be contained in the training set.}
%	\label{fig:reference-known-labels}
%\end{figure}

In our setting, the best we can hope for is one true positive superpixel in every frame. 
Choosing one true positive superpixel at random from the observed frames\footnote{the frames where the key was pressed in one of the gaze position observations}, and setting the others to negative / unlabeled, we can see in Figure \ref{fig:one-random-tp-per-frame} that the differences between our PU-loss and a standard exponential loss separating ``observed'' from ``unobserved'' superpixels is not very big\todo{is there a connection to the paper that showed it's only about thresholding???}, if we find a reasonable way to estimate the probabilities \todo{explain how this was done} of unlabeled superpixels being positive or negative. 
However, the standard classifier will strictly try to separate observed from non-observed superpixels, which is only the correct behaviour if the assumptions discussed in Chapter \ref{chap:background} are fulfilled -- if they are not fulfilled, it means that the learned classifier holds a bias (this has been discussed e.g.\ in \todoRef{give reference}). 
That this is truly happening in this case can, for example, be seen from the heat maps in Figure \ref{fig:bias-in-heatmaps}: Using our constructed PU-loss function, 0 becomes a reasonable threshold to distinguish between positive and negative samples, whereas for the other classifiers we would have to find a threshold below zero to get any positive labels in the output. 
For the eye tumor MRI, one issue with the evaluation was that there are only a few slices containing true positive information, and usually the user pressed the key during most of these frames. 
Therefore, evaluating only the generalization to the remaining frames is not a reasonable approach and, unlike in the instrument dataset, it will not tell us enough about the potential of segmenting the eye tumor. 
Instead, the evaluation has been done for a big part of the whole volume, including already seen superpixels during training. 
If we are later using only gaze observations, we will have to evaluate as well every superpixel of every frame.

\begin{figure}[ht]
	\centering
	\begin{subfigure}[h]{0.45\textwidth}
	\includegraphics[width=\textwidth]{d2-one_random_tp_per_frame-roc.pdf}	
		\caption*{instrument}
	\end{subfigure}
	~
	\begin{subfigure}[h]{0.45\textwidth}
	\includegraphics[width=\textwidth]{d2-one_random_tp_per_frame-pr.pdf}	
		\caption*{instrument}
	\end{subfigure}
	
	\vspace{3mm}
	\begin{subfigure}[h]{0.45\textwidth}
	\includegraphics[width=\textwidth]{d7-one_random_tp_per_frame-roc.pdf}	
		\caption*{eye tumor}
	\end{subfigure}
	~
	\begin{subfigure}[h]{0.45\textwidth}
	\includegraphics[width=\textwidth]{d7-one_random_tp_per_frame-pr.pdf}	
		\caption*{eye tumor (tested frames: [12:80])}
	\end{subfigure}	
	
	\vspace{3mm}
	\begin{subfigure}[h]{0.45\textwidth}
	\includegraphics[width=\textwidth]{d8-one_random_tp_per_frame-roc.pdf}	
		\caption*{cochlea}
	\end{subfigure}
	~
	\begin{subfigure}[h]{0.45\textwidth}
	\includegraphics[width=\textwidth]{d8-one_random_tp_per_frame-pr.pdf}	
		\caption*{cochlea (tested frames: [70:250])}
	\end{subfigure}		
	\caption{Randomly chosen true positive for each frame that was observed (i.e.\ where the user pressed the key) in one of the gaze observations.}
	\label{fig:one-random-tp-per-frame}
\end{figure}


\begin{figure}[ht]
	\centering
 	\begin{subfigure}[h]{0.48\textwidth}
	  \includegraphics[width=\textwidth]{d2-frame_00455-input}
	  \caption*{input image \\ (frame 455 of dataset ``instrument'')}
	\end{subfigure}
	~
	\begin{subfigure}[h]{0.48\textwidth}
	  \includegraphics[width=\textwidth]{d2-frame_00455-groundtruth}
	  \caption*{ground truth \\ \quad}
	\end{subfigure}
	
	\vspace{3mm}
	\begin{subfigure}[h]{0.495\textwidth}
	  \includegraphics[height=4.4cm]{d2-frame_00455-gradboost-heatmap}
	  \includegraphics[height=4.4cm]{d2-frame_00455-gradboost-colorbar}
	  \caption*{exponential loss \\ (gradient boosting)}
	  % shrinkage = 0.1, 5000 iterations, max tree depth = 2
	\end{subfigure}
	
	\vspace{3mm}
	\begin{subfigure}[h]{0.495\textwidth}
	  \includegraphics[height=4.4cm]{d2-frame_00455-svm-heatmap}
	  \includegraphics[height=4.4cm]{d2-frame_00455-svm-colorbar}
	  \caption*{SVM classifier \\ (rbf kernel with $\gamma = 0.0625$, $c = 10$)\\}
	\end{subfigure}	
	\begin{subfigure}[h]{0.495\textwidth}
	  \includegraphics[height=4.4cm]{d2-frame_00455-pugradboost-heatmap}
	  \includegraphics[height=4.4cm]{d2-frame_00455-pugradboost-colorbar}
		  \caption*{PU-losss \\ (gradient boosting)}
		  % 500 iterations, shrinkage = 0.1, stumps
	\end{subfigure}		
	\caption{}
	\label{fig:bias-in-heatmaps}
\end{figure}

\subsection{Labels inferred from gaze observations}
The difference between the above case of having one true positive per frame to our actual situation lies mainly in the distribution of the positive labels -- they are not, as assumed until now, randomly taken from all the positives of the observed frames, but instead they might often over-represent certain positives and under-represent others due to e.g.\ the fact that it is easier to fixate an edge than a smooth region in an image. 
Another issue that makes the problem harder is the present noise in the gaze observations: As discussed in Chapter \ref{chap:characterizing-gaze}, it is not true that each gaze positions indicates a true positive position in the image.
The naive approach performs clearly worse than ours because it does not distinguish between ... \todo{... stronger emphasis on whatever samples?} Experiments already done:


\begin{figure}[ht]
	\centering
	\begin{subfigure}[h]{0.45\textwidth}
	\includegraphics[width=\textwidth]{d2-gaze2-results-roc.pdf}	
		\caption*{instrument}
	\end{subfigure}
	~
	\begin{subfigure}[h]{0.45\textwidth}
	\includegraphics[width=\textwidth]{d2-gaze2-results-pr.pdf}	
		\caption*{instrument}
	\end{subfigure}
	
	\vspace{3mm}
	\begin{subfigure}[h]{0.45\textwidth}
	\includegraphics[width=\textwidth]{d7-gaze4-results-roc.pdf}	
		\caption*{eye tumor}
	\end{subfigure}
	~
	\begin{subfigure}[h]{0.45\textwidth}
	\includegraphics[width=\textwidth]{d7-gaze4-results-pr.pdf}	
		\caption*{eye tumor (tested frames: [12:80])}
	\end{subfigure}	
	
	\vspace{3mm}
	\begin{subfigure}[h]{0.45\textwidth}
	\includegraphics[width=\textwidth]{d8-gaze2-results-roc.pdf}	
		\caption*{cochlea}
	\end{subfigure}
	~
	\begin{subfigure}[h]{0.45\textwidth}
	\includegraphics[width=\textwidth]{d8-gaze2-results-pr.pdf}	
		\caption*{cochlea (tested frames: [70:250])}
	\end{subfigure}		
	\caption{Results using only actual gaze observations.}
	\label{fig:results-curves}
\end{figure}




%mainly because it means that a superpixel that has been labeled positive will have a multitude of negatively labeled ``opponents'' that look almost the same. It can be clearly seen that positions that have been observed only for a short time, even though they belong to the positive set, are not well separated from the negative set because very similar parts will end up in the negative set.



\todoWriteMore{contrast to what they did in 1st paper. Training labels according to real ground truth vs. training labels according to gaze position as suggested by Vilari\~no et al.:
...}




% Chapter 5
\chapter{Conclusion}
\label{chap:conclusion}
\section{Conclusion}
Gaining positive labels using eye-tracking is a realistic idea and it can be used to segment images by interpreting the problem as a semi-supervised problem with only positive and unlabeled training samples. The smaller the structures in the volume / video, the more difficult it is to get reliable positive data. \todo{Mention also rate of change instead of only size} We showed that the previous approach of Vilari\~no et al.\ is not directly applicable to general data and reformulated the underlying problem to overcome the restrictions. For synthetic as well as for real data, our interpretation as a PU-learning problem improves the classification results. The suggested loss function can be used within a standard gradient boosting framework. 
A drawback of our approach is that providing the correct probabilities to define the loss function is not necessarily a trivial task. We used a combination of distances in spatial and feature domain combining information of the current frame and the whole video sequences. 


\section{Future work}

\subsection{Feature Learning}
The features we used for this work were basic hand-engineered ones. 
A strategy that might lead to better descriptors might be to learn features in an unsupervised way. 
For example \cite{ng2011sparse} describes nicely how to use sparse autoencoders to learn features from unlabeled data. This strategy was shown to be helpful in training deep neural networks without getting stuck in local minima \cite{bengio2007greedy}.

\subsection{Use Consistency}
The data we are dealing with in this application are usually smoothly changing over time.
While we did consider information from the whole sequences / volumes by including it in our probabilities (section \ref{sec:real-data}), smoothness constraints could possibly be used to identify gaze positions that are unlikely to indicate true positive data. 
Considering supervoxels instead of superpixels could further improve in using the consistencies along the third dimension. Supervoxels have previously been used for video segmentation for example by Corso et al.\ \cite{CoShDuTMI2008} and Grundmann et al.\ \cite{grundmann2010efficient}.



%END Doc
%-------------------------------------------------------

\bibliography{thesis}
\bibliographystyle{plain}

\end{document}
