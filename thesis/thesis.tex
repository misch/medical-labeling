
\documentclass[oneside,a4paper]{book}
%\pagestyle{headings}

\input{preamble}

% A B S T R A C T
% % % % % % % % % % % % % % % % % % % % % % % % % % % % % % % % % %
\chapter*{\centering Abstract}
\begin{quotation}
\noindent 
Clinical decisions are nowadays often based on medical imaging and the desire for automated systems to support clinicians is high. 
Yet, using supervised learning techniques requires a vast amount of pre-annotated training data.
Producing annotations in medical images is a time-consuming task that requires experienced experts and is a critical bottleneck in medical image analysis. 
We discuss a previous attempt to use eye-tracking to produce ground truth data and we show the difficulties that come with this great potential for time cost reduction. 
We overcome the limitations by reformulating the underlying problem as a PU-learning problem and suggest an intuitive PU-loss function that can be optimized using gradient boosting.
Our approach outperforms the other discussed approach on real and synthetic data and brings us a step towards automated segmentation using eye-tracking.

Prof.\ Dr.\ Paolo Favaro, Computer Vision Group, Institut f\"ur Informatik, Universit\"at Bern, Supervisor

Dr. Raphael Sznitman, Ophthalmic Technology Laboratory, ARTORG Center for Biomedical Engineering Research, Universit\"at Bern, Co-Supervisor
\end{quotation}
\clearpage


% C O N T E N T S 
% % % % % % % % % % % % % % % % % % % % % % % % % % % % % % % % % % % % % % % %
\tableofcontents


% Chapter 1
\chapter{Introduction}
\label{chap:introduction}

\section{Motivation}
      Medical imaging is a set of techniques to produce visual representations of the interior of a body. 
      Such images are used for clinical analysis and medical intervention, e.g. to diagnose and treat disease. 
      Techniques like computed tomography or magnetic resonance imaging reveal internal structures hidden by skin and bones. 
      Clinical decisions are nowadays often based on these data, and the desire for automated systems to support the data analysis is high. 
      Yet, the data situation in medicine is not satisfying; we have in general large amounts of data produced by medical devices, but it cannot straight away be used to create automated systems to support clinicians in their everyday work, that is, for diagnosis or even treatment purposes. 

    \begin{enumerate}
      \item not enough data for training supervised methods; that's what we wanna change
      \item gaze: lower time bound
    \end{enumerate}
  
  
\section{Goal}
Produce reliable ground truth data for future training while achieving the lower bound of time given by experts seeing a structure of interest.

\section{Overview}

% Chapter 2
\chapter{Background}
\label{chap:background}
\todoWriteMore{1st paper and its limitations (my own ideas: add also things about: supervised learning, tree models, boosting, gradient boosting, loss functions in gradient boosting, PU formulation, loss functions in gradient boosting, PU formulation, loss functions for PU problems (see paper)}
This chapter will briefly explain the method of Vilari\~no et al. to use gaze-tracking for polyp detection as well as elaborate their assumptions, limitations and conceptual problems that arise when trying to generalize their methods to arbitrary data gained through medical imaging methods. Further, there will be given a brief overview and explanation of the used techniques during this project.

\section{Classifiers and Boosting}
In classification, the task is to find for a given data sample $x \in \mathbb{R}^n$ a class label $f(x) \in \{c_1, c_2, ..., c_k\}$. As an example, $x$ could be a representation of an e-mail and $\{c_1,c_2,c_3\}$ could be \{``spam'', ``business'', ``private''\}. For simplicity, usually a binary classifier with $f(x) \in \{-1,1\}$ is considered and multiclass problems are later formulated in terms of the two-class approach. \todo{mention margins} 
Finding the function $f^*$ that optimally assigns positive and negative class labels to all inputs has been one of the big focuses in research within the Machine Learning and Computer Vision community. There are different approaches to handle different situations: Supervised learning methods depend on the availability of labeled training data, whereas unsupervised methods aim at making sense of the data without any examples. 
Vilari\~no et al.\ used gaze-data to directly generate training examples for the widely used Support Vector Machine (SVM) classifier to show some first promising results. 
This classifier optimizes a linear\footnote{kernels...}\todo{footnote} boundary between the classes of a given training set. Instead of directly using the gaze positions as training set for a fully supervised method like a SVM classifier, we will investigate the meta-problem of creating correct training data from gaze positions for later usage in supervised settings. 
We formulate this meta-problem as a classification-/segmentation task itself and put it into the context of a semi-supervised setting. 
Whereas there have been suggestions to extend SVM classifiers for usage in semi-suervised settings \todoRef{add ref}, we are going to use a Gradient Boosting algorithm that allows more flexibility in what is optimized, and will enable us to formulate the optimization goal in an intuitive way, taking into consideration the fact that for a big part of the given data, we don't know the labels.

The idea of boosting is that while a simple ``weak'' classifier might produce predictions that are only slightly better than random guessing, combining several weak classifiers could produce a powerful committee. 
Boosting sequentially applies the weak classifier (e.g.\ decision trees or neural nets) to modified versions of the original input data. 
The modification of the input data in each step depends on the previously generated classifier: Observations that were misclassified by the previous classifier get higher weights and the weights for already correctly classified observations are decreased. 
A very popular boosting algorithm is the so-called AdaBoost algorithm introduced by Freund and Schapire in 1997 \todoRef{add to bibliography: Yoav Freund and Robert E. Schapire. A decision-theoretic generalization of on-line learning and an application to boosting}. 
From a statistical point of view, this algorithm minimizes the exponential loss of the margin, a loss function that gives a very high penalty to negative margins (i.e.\ wrongly classified samples),$\sum_i e^{-y_i f(x_i)}$. 
It turned out that this is not always the desired thing to do; especially if the training data contains outliers, concentrating on the correct classification of those rather leads to a bad generalizatin error and it might be better to tolerate the outlier to be classified wrongly \todo{maybe make a figure}. 
That AdaBoost can be interpreted as an optimization algorithm (more precisely, forward stagewise additive modeling) based on exponential loss was discovered only later, and with it the desire to develop simple feasible boosting algorithms for arbitrary loss functions. 
However, for arbitrary (convex, differentiable) loss functions it is not trivial to solve the optimization problem arising in each step of the mentioned forward stagewise additive modeling approach: 
In each step, what has to be found are the parameters for the optimal weak classifier (e.g.\ a decision tree) that minimizes the loss function, if added to the current model. 
For exponential loss, this simplifies to a weighted exponential criterion for the new tree and a greedy recursive-partitioning algorithm can be used with this weighted exponential loss as a splitting criterion. 

On the other hand, finding the unconstrained minimum of an arbitrary differentiable convex (loss-)function could be done via numerical optimization methods such as steepest gradient descent. 
However, in our setting, the therefore needed gradients are defined only on the training data points, and we have the constraint that tree-components, unlike the negative gradient components used in plain steepest descent, have to be predictions of a decision tree instead of the unconstrained maximal descent direction. 
A way to get a predictive model (i.e.\ a model that generalizes to previously unseen data, which is the ultimate goal) is, to induce a basic classifier (decision tree) at each iteration that is fit to the calculated gradients. 
The here summarized descriptions are a short version of a much more detailed explanation in [book...]\todoRef{add and check ref...}.

\section{Build a classifier using gaze-tracking}
In 2007, Vilari\~no et al. published a method that used gaze-tracking for polyp detection ( \todoRef{reference...}). The core idea of Vilari\~no et al. was to train a classifier using expert's gaze positions to generate positive and negative training samples, instead of manually labeled training data. 
The availability of labeled training data is the bottleneck for enabling Machine Learning and Computer Vision methods for clinical applications. 
The idea of Vilari\~no et al. has a great potential to tackle this issue because it is aimed at reducing the time costs to create training data for classifiers. 

However, the application of Vilari\~no et al. was limited to polyp detection in colonoscopy videos and theirway to build the training set from gaze observations implicitely used the following assumptions:
\begin{enumerate}
 \item in each video frame, there was at most one structure of interest (polyp)
 \item a structure of interest never exceeded the size of $128 \times 128$ pixels
\end{enumerate}
Examples of how their data looked like are shown in Figure \ref{fig:vilarinoPolypExamples}.\todoRef{ref in Caption}

\begin{figure}[ht]
	\centering
	\includegraphics[width=\textwidth]{vilarino-polyp-examples}
	\caption{some examples of image patches containing polyps in Vilari\~no et al.'s work. The pictures are taken from [??]}.
	\label{fig:vilarinoPolypExamples}
\end{figure}

Using datasets that fulfill assumptions 1 and 2 from above, we could achieve reasonable results with Vilari\~no's approach regarding detection (Figure \ref{fig:theirapproachairplane}). As we are aiming for a pixel-/voxelwise segmentation of our input, we have to capture the object boundaries and regarding this task, we could significantly improve the quality of the output by using SLIC superpixels \todo{perhaps specify} instead of the values of pre-processed image patches to describe image regions. 
The results are visible in Figure \ref{fig:airplaneSLIC}. [\todo{reason with data (compare SVM and grad boost)!}To get a useful binary result, we could not use all of the negatively labeled superpixels because the SVM classifier could not handle the class imbalance, as can be seen from the heat map in Figure \ref{fig:airplaneSVMclassImbalance}: Even though the aircraft clearly gets higher scores than the background, they remain below zero. Therefore we decided to work with a gradient boosting technique as described in Chapter \ref{chap:background}].

\begin{figure}[ht]
	\centering
	\begin{subfigure}[h]{0.31\textwidth}
		\includegraphics[width=\textwidth]{airplane-input-frame_00189}	
		%\missingfigure[figwidth=\textwidth]{airplane video input (several frames)}
		\caption*{input (frame 189)}
	\end{subfigure}
	~
	\begin{subfigure}[h]{0.31\textwidth}
		\includegraphics[width=\textwidth]{airplane-binaryOutput-frame189-svm-patches-c10}	
		%\missingfigure[figwidth=\textwidth]{results (patch vs. superpixels}
		\caption*{binary output}
	\end{subfigure}
	~
	\begin{subfigure}[h]{0.31\textwidth}
		\includegraphics[width=\textwidth]{airplane-heatmapOutput-frame189-svm-patches-c10}	
		%\missingfigure[figwidth=\textwidth]{results (patch vs. superpixels}
		\caption*{heat map}
	\end{subfigure}
	
	\vspace{3mm}
	\begin{subfigure}[h]{0.31\textwidth}
		\includegraphics[width=\textwidth]{airplane-input-frame_00249}	
		%\missingfigure[figwidth=\textwidth]{airplane video input (several frames)}
		\caption*{input (frame 249)}
	\end{subfigure}
	~
	\begin{subfigure}[h]{0.31\textwidth}
		\includegraphics[width=\textwidth]{airplane-binaryOutput-frame249-svm-patches-c10}	
		%\missingfigure[figwidth=\textwidth]{results (patch vs. superpixels}
		\caption*{binary output}
	\end{subfigure}	
	~
	\begin{subfigure}[h]{0.31\textwidth}
		\includegraphics[width=\textwidth]{airplane-heatmapOutput-frame249-svm-patches-c10}	
		%\missingfigure[figwidth=\textwidth]{results (patch vs. superpixels}
		\caption*{heat map}
	\end{subfigure}	
	\caption{inputs and outputs obtained with Vilari\~no's approach. For the shown examples, we used a SVM classifier ({\tt libsvm} package for MATLAB) with a RBF kernel ($\gamma = 0.625$) and a rather high regularization value of $c = 10$. Note that the region containing the airplane was at least partly detected.}
	\label{fig:theirapproachairplane}
\end{figure}

\begin{figure}[ht]
	\centering
	\begin{subfigure}[h]{0.31\textwidth}
		\includegraphics[width=\textwidth]{airplane-input-frame_00189}
		%\missingfigure[figwidth=\textwidth]{airplane video input (several frames)}
		\caption*{input (frame 189)}
	\end{subfigure}
	~
	\begin{subfigure}[h]{0.31\textwidth}
		\includegraphics[width=\textwidth]{airplane-binaryOutput-frame189-svm-superpixelsColor-c10}	
		%\missingfigure[figwidth=\textwidth]{results (patch vs. superpixels}
		\caption*{binary output}
	\end{subfigure}
	~
	\begin{subfigure}[h]{0.31\textwidth}
		\includegraphics[width=\textwidth]{airplane-heatmapOutput-frame189-svm-superpixelsColor-c10}	
		%\missingfigure[figwidth=\textwidth]{results (patch vs. superpixels}
		\caption*{heat map}
	\end{subfigure}
	
	\vspace{3mm}
	\begin{subfigure}[h]{0.31\textwidth}
		\includegraphics[width=\textwidth]{airplane-input-frame_00249}	
		%\missingfigure[figwidth=\textwidth]{airplane video input (several frames)}
		\caption*{input (frame 249)}
	\end{subfigure}
	~
	\begin{subfigure}[h]{0.31\textwidth}
		\includegraphics[width=\textwidth]{airplane-binaryOutput-frame189-svm-superpixelsColor-c10}	
		%\missingfigure[figwidth=\textwidth]{results (patch vs. superpixels}
		\caption*{binary output}
	\end{subfigure}	
	~
	\begin{subfigure}[h]{0.31\textwidth}
		\includegraphics[width=\textwidth]{airplane-heatmapOutput-frame189-svm-superpixelsColor-c10}	
		%\missingfigure[figwidth=\textwidth]{results (patch vs. superpixels}
		\caption*{heat map}
	\end{subfigure}	
	\caption{inputs and outputs obtained with color-based features of SLIC superpixels. We used the same classifier as in Figure \ref{fig:theirapproachairplane}, but used only a subset of the negatively labeled superpixels for training, as the SVM could not handle the class imbalance (see Figure \ref{fig:airplaneSVMclassImbalance}). }
	\label{fig:airplaneSLIC}
\end{figure}

\begin{figure}[ht]
	\centering
	\includegraphics[width=\textwidth]{airplane-heatmapOutput-frame189-svm-c10-wholeTrainingSet}
	\caption{heat map of the resulting scores with a SVM trained on all the labeled samples }
	\label{fig:airplaneSVMclassImbalance}
\end{figure}


Datasets that do not fulfill both assumptions cause conceptual and practical problems. In the ideal case we have, for each frame, one true positive patch / superpixel given by the recorded gaze position. However, the assumption that all the other non-overlapping patches / superpixels are negative, is not valid, if one of the above-mentioned assumptions is not fulfilled. Unfortunately, for many datasets those are not valid assumptions. Figure \ref{fig:nonValidAssumption} shows a dataset containing a surgical instrument with a bigger extent. It is clearly visible that the negatively labeled patches / superpixels are not necessarily true negatives.



\begin{figure}[ht]
	\centering
	\begin{subfigure}[h]{0.48\textwidth}
		\includegraphics[width=\textwidth]{dataset2gazePositionFrame207}
		%\missingfigure[figwidth=\textwidth]{airplane video input (several frames)}
		\caption*{frame 207 (red: gaze position)}
	\end{subfigure}
	~
	\begin{subfigure}[h]{0.48\textwidth}
	    \includegraphics[width=\textwidth]{dataset2SLICsegmentationFrame207}
	    %\missingfigure[figwidth=\textwidth]{airplane video input (several frames)}
	    \caption*{SLIC superpixels (red: gaze position)}
	\end{subfigure}
	
	\vspace{3mm}
	\begin{subfigure}[h]{0.31\textwidth}
		\includegraphics[width=\textwidth]{dataset2positivePatchFrame207}	
		%\missingfigure[figwidth=\textwidth]{results (patch vs. superpixels}
		\caption*{positive patch}
	\end{subfigure}
	~
	\begin{subfigure}[h]{0.31\textwidth}
		\includegraphics[width=\textwidth]{dataset2positiveSuperpixelFrame207}	
		%\missingfigure[figwidth=\textwidth]{results (patch vs. superpixels}
		\caption*{positive superpixel}
	\end{subfigure}
	
	\vspace{3mm}
		\begin{subfigure}[h]{0.48\textwidth}
		\includegraphics[width=\textwidth]{dataset2negativePatchesFrame207}	
		%\missingfigure[figwidth=\textwidth]{airplane video input (several frames)}
		\caption*{``negative'' patches}
	\end{subfigure}
	~
	\begin{subfigure}[h]{0.48\textwidth}
		\includegraphics[width=\textwidth]{dataset2negativeSuperpixelsFrame207}	
		%\missingfigure[figwidth=\textwidth]{results (patch vs. superpixels}
		\caption*{``negative'' superpixels}
	\end{subfigure}	
	\caption{blah blah yaddah yaddah...). }
	\label{fig:nonValidAssumption}
\end{figure}


If we consider one $128 \times 128$-patch / superpixel positive and all the other non-overlapping ones as negative examples, we run into mainly 2 issues: 
\begin{enumerate}
 \item unbalanced dataset (few positives, many negatives)
 \item very high likelihood that each positive sample has a corresondance in the negative training set.\todo{rewrite}
\end{enumerate}
This leads to the optimal solution being a negative label for each test sample.

A natural step is therefore to formulate the problem in a different way, namely not in ``separate positive from negative samples'', but instead as ``given some positive samples, figure out whether or not the other, unlabeled ones, also belong to the positive or instead to the negative class.'' \todo{maybe illustration?}
\todo{somewhere here... split in chap 2 and chap 4; chap 4 should start only here!}
This so-called PU-problem recently got a lot of attention and applications in document classification and will be discussed in Chapter \ref{chap:chap:learning-with-unlabeled-data}.

Their training data: 80\% of the ``gaze-labeled'' ROIs. Their test data: 20\% of the ``gaze-labeled'' ROIs. They didn't compare to any ``real'' ground truth, but basically only evaluated how well the SVM separates ``observed'' vs. ``not observed'' regions \todo{how do we do in this case?} -- however, this evaluation does not take into account that the generated labels from the gaze positions are inherently noisy.
\todoWriteMore{mention their actual results}
%
% Chapter 3
\chapter{Characterizing Gaze}
\label{chap:characterizing-gaze}
In this chapter we describe the used setup to get the gaze observations and assess the quality of extracted positive labels gained from the observed gaze positions. Of special interest is the question how reliable our data are with respect to the assumption that gaze observation naturally provide us with positive labels for identifying a structure of interest.

\section{Setup}
We used an affordable eye-tracking device called ``The Eye Tribe''. 
According to the producer's website (\url{http://dev.theeyetribe.com/general/}) the device has an accuracy of at least 1 degree visual angle. We placed the eye-tracker on a tripod below a 32.4 cm\,$\times$\,51.8 cm screen and observed the screen from a distance of approximately 60 cm (Figure \ref{fig:theeyetribe} illustrates the physical setup).
\begin{figure}[ht]
	\centering
	\includegraphics[width=\textwidth]{theeyetribe}	
	\caption{A user in front of The Eye Tribe. The eye-tracker is placed on a tripod below the screen and observed from a distance of $\approx$60 cm. The image is taken from \url{http://dev.theeyetribe.com/general/}).}
	\label{fig:theeyetribe}
\end{figure}
In this setup, the minimum accuracy of 1 degree corresponds to an on-screen error of approximately 1 cm. 
Figure \ref{fig:onedegreecircle} gives an idea of how big this error is compared to the structures in our datasets. 
The instrument dataset is, compared to the cochlea CT scan much bigger in size and therefore the error is less likely to draw the measurement off the object, if the true gaze position is focused on the object.

\begin{figure}[ht]
	\centering
	\begin{subfigure}[h]{0.31\textwidth}
		\includegraphics[width=\textwidth]{one-degree-circle-cochlea-17pix-frame195_small_new}	
		\caption*{cochlea}
	\end{subfigure}
	~
	\begin{subfigure}[h]{0.31\textwidth}
		\includegraphics[width=\textwidth]{one-degree-circle-instrument-18_62pix-frame195_small_new}
		\caption*{instrument}
	\end{subfigure}
	~
	\begin{subfigure}[h]{0.31\textwidth}
		\includegraphics[width=\textwidth]{one-degree-circle-eyeMRI-12_28pix-frame46_small}	
		\caption*{eye tumor}
	\end{subfigure}
	\caption{Visual illustration of how much error a 1 degree visual angle causes in the different datasets using our described setup. Whereas most parts of the instrument are big in size, the fine structures in the images of the cochlea or the eye tumor might not be hit by the measured gaze position, even when an expert is looking exactly at them.}
	\label{fig:onedegreecircle}
\end{figure}

Like Vilari\~no et al.\ \cite{vilarino2007automatic}, we asked the user to press and hold a key to select video sequences containing a structure of interest, and to keep their focus on the actual object. 
Therefore, what is presented in this work, is an active application of eye-tracking with the potential to become passive in the future.

\section{Reliability of Extracted Positives}
In this section, we evaluate whether the assumption of Vilari\~no et al.\ is valid that the observer always indicates true positive image regions by pressing a key. 
A ``positive image region'' is in this context simply a pixel on the screen that contains the object of interest.
We did an experiment with a simple image that should be free from distractions and where it should be easy for the user to focus on the target. 
Then we continued the evaluation on 3 real datasets for which we have manually segmented ground truth data:
\begin{itemize}
 \item Dataset ``instrument'': A video sequence of a surgical instrument during an endoscopy.
 \item Dataset ``eye tumor'': A 3D MRI volume of an eye tumor.
 \item Dataset ``cochlea'': A 3D CT volume of the cochlea (inner ear).
\end{itemize}

\noindent The video sequences were played on full screen at 30 frames per second, and the 3D scans were played as a video at 10 frames per second. 

As the user had the task to focus on the object in the videos, it is expected that the majority of recorded gaze positions are located at true positives, or that at least some true positive points can be found within a 1 degree visual angle of the recorded gaze position. 
To separate the human error from the measurement error we investigated for each of the real datasets the actual distance between the gaze position and the closest true positive point. 
Values above 1 degree visual angle mean human error whereas values below could mean both, human error or measurement error. 

\subsection{Fixating a Simple Shape}
To get a first idea of the actual measurement error, the gaze positions were measured and evaluated for a very simple case. 
A video of a non-moving black point on white background in the middle of the screen should be free from distractions and give us a first hint about the measurement error. 
Two measurements were taken. 
Figure \ref{fig:gazeMeasurementAccuracy} shows the eye movements in 2-dimesional space (left Figure) as well as it shows the distance to the black dot over time (right Figure). 
The errors remain well below 1 degree, except in the case of blinking. 
The distance to the center of the image (black dot) is on average between 5 and 7 pixels. 
This corresponds to a visual angle of approximately 0.25 to 0.4 degrees. 
Therefore, giving a tolerance of 0.4 degrees seems to be a necessary step to account for the inaccuracy of the eye-tracking device.

\begin{figure}[ht]
	\centering
	\begin{subfigure}[h]{0.41\textwidth}
	      \setlength{\fboxsep}{0pt}%
	      \setlength{\fboxrule}{0.5pt}%
	      \centering
	      \fbox{\includegraphics[width=\textwidth]{gazeMeasurementAccuracy2D.pdf}}
	\end{subfigure}
	~
	\begin{subfigure}[h]{0.48\textwidth}
		\includegraphics[width=\textwidth]{gazeMeasurementAccuracy1D.pdf}	
		%\caption{}
	\end{subfigure}
	\caption{For the task of staring at the black dot in the middle of the screen, the errors remain well below 1 degree. Blinking causes considerable outliers. The mean distance to the actual center of the image (dotted blue and green lines) is between 5 and 7 pixels in this test video. This corresponds to a visual angle of approximately 0.25 to 0.4 degrees.}
	\label{fig:gazeMeasurementAccuracy}
\end{figure}

\subsection{Fixating Real Objects}
To test the gaze observation on real data, we used the above described three datasets. 
The top three plots of Figures \ref{fig:distanceToClosestPositiveD2}, \ref{fig:distanceToClosestPositiveD7} and \ref{fig:distanceToClosestPositiveD8} show for each dataset the distances to the closest true positive pixel for three different recorded gaze sequences. The fourth plot of those Figures shows the object size in each frame.
The instrument dataset (Figure \ref{fig:distanceToClosestPositiveD2}) shows promising results: 
In the three recorded gaze sequences, the majority of the values are exactly zero, indicating that the gaze hits the actual object. 
Measurement errors do not seem to influence the potential accuracy of our method here, as the cases where the gaze is not on the object are clear human-caused outliers with a distance of more than 1 degree visual angle.
A similar conclusion holds for the eye tumor dataset (Figure \ref{fig:distanceToClosestPositiveD7}). 
The gaze observations show that between frames 40 and 50, when the tumor is big in size, it is well hit by the gaze. 
The rather big distances to any positive pixels before and after this interval might reflect the fact that the size of the structure of interest decreases rapidly in these frames (see the last plot in Figure \ref{fig:distanceToClosestPositiveD7}). 
The recordings from the cochlea dataset (Figure \ref{fig:distanceToClosestPositiveD8}) show that it is a rather rare event in this dataset that the gaze really hits the object. 
However, the distance to the true positives stays mostly below 1 degree visual angle. This suggests that in this case, the inaccuracy of the device might be a problem.
Yet, considering the previously discussed measurements of the non-moving black dot, it is more likely that the errors are due to the very fine structures to detect in this dataset. Like in the previous case of the eye tumor, the gaze accuracy decreases when the size of the object decreases. 
It seems that not only the object size is of importance when trying to fixate positive points, but also the change rate of the size. 
In all the three datasets, but best visible for the eye tumor and cochlea, the gaze tends to drift away from the object whenever the size of the object decreases, whereas an increasing size seems to attract the attention of the observer and make it easier to fixate the object. 
This suggests that the reliability of the collected gaze sequences could be improved by playing the video backwards or giving the user full control of the playback mode.

The discussed plots show that the cochlea dataset can be considered the most challenging one used during this project. 
It is the one where the gaze sequences are the least focused on the object. 

The rate of change in size, and also the absolute size seem to be critical when trying to focus an object throughout a sequence of images. 
Figure \ref{fig:relativeobjectsize} and Table \ref{tab:avgobjectsize} show comparisons of the object sizes. 
Because the relative size\footnote{The relative object size was calculated with respect to the resolution of the used images.}, of the instrument is much bigger than the size of the eye tumor or cochlea, the likelihood to randomly find a positive pixel is higher for the instrument dataset than for the other two. Therefore, it is likely that for the instrument dataset the final results will be less prone to little errors in the gaze than for the other two. 
\begin{figure}[ht]
	  \includegraphics[width=\textwidth]{closestPositiveDataset2vid2.eps}
	  \includegraphics[width=\textwidth]{closestPositiveDataset2vid5.eps}
	  \includegraphics[width=\textwidth]{closestPositiveDataset2vid4.eps}
	  
	  \vspace{3mm}
	  \includegraphics[width=\textwidth]{size_instrument.eps}
	  \caption{Instrument dataset. Top three plots: distances to the closest true positive pixel for three different recorded gaze sequences. Fourth plot: object size. As expected, many values are exactly zero which means that hitting the object is rather easy in this dataset. Where the gaze is not on the object, the typically big outlier values in the distance indicate that this is not due to a measurement error, but instead the eye was really not on the object.}
	\label{fig:distanceToClosestPositiveD2}
\end{figure}

\begin{figure}[ht]
	  \includegraphics[width=\textwidth]{closestPositiveDataset7vid1.eps}
	  \includegraphics[width=\textwidth]{closestPositiveDataset7vid4.eps}
	  \includegraphics[width=\textwidth]{closestPositiveDataset7vid6.eps}
	  
	  \vspace{3mm}
	  \includegraphics[width=\textwidth]{size_eyetumor.eps}	  
	  \caption{Eye tumor dataset. Top three plots: distances to the closest true positive pixel for three different recorded gaze sequences. Fourth plot: object size. The small number of available values means that, in general, the structure of interest (tumor) is small in size (not available values mean that there are no positive values in the ground truth frame). Between frames 40 and 50 the tumor seems to be well visible and big enough in size to be rather reliably hit by the gaze. Note that in frames 50 -- 54 the size of the tumor is still considerably big, but the gaze positions are getting off the object already from frame 50 on. This indicates that not only the actual size but also the change rate of size has an effect on the quality of the gaze positions.}
	\label{fig:distanceToClosestPositiveD7}
\end{figure}

\begin{figure}[ht]
	  \includegraphics[width=\textwidth]{closestPositiveDataset8vid3.eps}
	  \includegraphics[width=\textwidth]{closestPositiveDataset8vid6.eps}
	  \includegraphics[width=\textwidth]{closestPositiveDataset8vid7.eps}
	  
	  \vspace{3mm}
	  \includegraphics[width=\textwidth]{size_cochlea.eps}	  
	  \caption{Cochlea dataset. Top three plots: distances to the closest true positive pixel for three different recorded gaze sequences. Fourth plot: object size. In this dataset, it is rather rare that the object of interest (cochlea) is actually hit by the measured gaze positions; only in few cases the distance is exactly zero. The high variation in distance to the object indicates that it is hard to follow the fine structures with the eyes, and probably only a small part of the error might be caused by inaccuracy in the measurements. Note that as in Figure \ref{fig:distanceToClosestPositiveD7}, we can observe that the rate of change in object size is related to the distance of gaze positions to true positive pixels. Especially the part between frame 150 and 200 is interesting. Even the little period (frames 173-187) of size fluctuations before the peak at frame 192 seems to cause considerable increases (but still below 1 degree) in the distance plots above.}
	\label{fig:distanceToClosestPositiveD8}
\end{figure}

\begin{figure}[ht]
	\centering
	  \includegraphics[width=\textwidth]{size_relative.png}
	\caption{Object areas relative to the resolution of the image frames. The small relative area of the eye tumor and the cochlea in the scans may explain the difficulties to reliably hit the object with the gaze.}
	\label{fig:relativeobjectsize}
\end{figure}

\begin{table}[ht]
	\centering
	  \caption{A listing of the sizes of the structures of interest, averaged over all the frames of a dataset that contain positive information.}
	  \label{tab:avgobjectsize}
	\begin{tabular}{ | c  c  c  c | }
	\hline
				& instrument 	& eye tumor & cochlea \\ \hline
	  resolution  		& $576 \times 720$ & $380 \times 384$ & $526 \times 429$ \\ 
	  average area [px]	& $39378$ 	& $981.12$ 	 & $605.86$ \\
	  average rel. area [\%]& $9.5$ 		& $0.67$ 		 & $0.2685$ \\ \hline
	\end{tabular}
\end{table}

\subsection{Quality of Extracted Superpixels}
So far we have considered possible causes for the gaze positions not to be focused on the object. 
We will now briefly focus on the effects of that problem with respect to our later task of using the gaze positions to generate positive training samples. 
To check how much true positive / false positive information we are actually using for our work, we visualized the amount of positive information in the following way: 
For each frame where the user pressed a key, we extracted the superpixel that contains the corresponding gaze position. 
Each of the extracted superpixels contains many pixels and we calculated the relative amount of true positive pixels among them. 
A reason for those superpixels containing less than 100\% true positive pixels is that some of the extracted superpixels describe a region that does not contain the object at all. 
This would lead to a fraction of 0\% true positives. 
The other reason is that the gaze actually hits the object, but the corresponding superpixel contains other information than just the object, or that the gaze does not hit the object, but the superpixel includes parts of it. 
For the same three gaze sequences than before, the relative amount of positive information in each extracted superpixel is shown in Figures \ref{fig:positiveFractionD2}, \ref{fig:positiveFractionD7} and \ref{fig:positiveFractionD8}. 
We can see in Figure \ref{fig:positiveFractionD2} that even a small distance to the actual object can cause considerable noise in the extracted superpixels. 
For example for the first sequence (top row in Figure \ref{fig:positiveFractionD2}, there are some extracted superpixels containing less than 50\% positive information around frame 400, even though Figure \ref{fig:distanceToClosestPositiveD2} shows that the distance between the gaze and the actual object was greater than zero during only 5 frames and never greater than 2.5 pixels. On the other hand, the effect of a similar case in the second sequence around frame 320 is not so dramatic and in total, the extracted superpixels contain more true positive than false positive information, which is a promising precondition because we aim at using them as reliable positive training samples. 

For the eye tumor and the cochlea data, the situation is more critical. Figures \ref{fig:positiveFractionD7} and \ref{fig:positiveFractionD8} show that there is a large amount of wrong information in the extracted positive superpixels. In fact, many of the extracted superpixels contain less than 50\% of true positive information. For both datasets, even quite some superpixels contain $0\%$ true positive information.

It is not a surprising observation that the amount of positive information is low when the gaze is slightly off the object. As the used SLIC superpixels respect edges, the superpixel containing a position that is only slightly off the object will typically not contain any positive information. We considered that increasing the superpixel size might help to avoid this problem, but found that this is not true. The amount of positive information for slightly off gaze positions does not necessarily change with the superpixel size, as the examples in Figure \ref{fig:gazeOffSuperpixelSize} show. 
 

\begin{figure}[ht]
	  \includegraphics[width=\textwidth]{fractionsDataset2vid2.eps}
	  \includegraphics[width=\textwidth]{fractionsDataset2vid5.eps}	  
	  \includegraphics[width=\textwidth]{fractionsDataset2vid4.eps}
	  
	  \centering
	  \includegraphics[width=\textwidth]{fraction-legend.eps}
	  \caption{Instrument dataset: The extracted superpixels contain in total more true positive (fraction $> 50\%$) than false positive (fraction $< 50\%$) information, which is good because we want to use them as reliable positive training samples. In average, the extracted superpixels contain $\approx 80\%$ true positive information (blue dotted lines).}
	\label{fig:positiveFractionD2}
\end{figure}

\begin{figure}[ht]
	  \includegraphics[width=\textwidth]{fractionsDataset7vid1.eps}
	  \includegraphics[width=\textwidth]{fractionsDataset7vid4.eps}
	  \includegraphics[width=\textwidth]{fractionsDataset7vid6.eps}
	  
	  \centering
	  \includegraphics[width=\textwidth]{fraction-legend.eps}
	  \caption{Eye tumor dataset: There is a large amount of wrong information in the extracted positive superpixels. Many of the extracted superpixels contain less than 50\% of true positive information. The superpixels that were extracted using recorded gaze positions contain between 28 and 35\% true positive information in average (blue dotted lines).}
	\label{fig:positiveFractionD7}
\end{figure}

\begin{figure}[ht]
	  \includegraphics[width=\textwidth]{fractionsDataset8vid3.eps}
	  \includegraphics[width=\textwidth]{fractionsDataset8vid6.eps}
	  \includegraphics[width=\textwidth]{fractionsDataset8vid7.eps}
	  
	  \centering
	  \includegraphics[width=\textwidth]{fraction-legend.eps}
	  \caption{Cochlea dataset: As in Figure \ref{fig:positiveFractionD7}, there is a large amount of wrong information in the extracted positive superpixels. Many of the extracted superpixels contain  $0\%$ true positive pixels. They contain only between 16 and 29\% true positive information in average (blue dotted lines).}
	\label{fig:positiveFractionD8}
\end{figure}

\begin{figure}[ht]
	\centering
	\begin{subfigure}[h]{0.31\textwidth}
	      \includegraphics[width=\textwidth]{superpixelSize1instrument}
	\end{subfigure}
	~
	\begin{subfigure}[h]{0.31\textwidth}
		\includegraphics[width=\textwidth]{superpixelSize2instrument}	
		%\caption{}
	\end{subfigure}
	~
	\begin{subfigure}[h]{0.31\textwidth}
		\includegraphics[width=\textwidth]{superpixelSize3instrument}	
		%\caption{}
	\end{subfigure}	
	
	\vspace{3mm}
	\begin{subfigure}[h]{0.31\textwidth}
	      \includegraphics[width=\textwidth]{superpixelSize1eye_20px}
	\end{subfigure}
	~
	\begin{subfigure}[h]{0.31\textwidth}
		\includegraphics[width=\textwidth]{superpixelSize2eye_30px}	
		%\caption{}
	\end{subfigure}
	~
	\begin{subfigure}[h]{0.31\textwidth}
		\includegraphics[width=\textwidth]{superpixelSize3eye_100px}	
		%\caption{}
	\end{subfigure}	
	\caption{Extending the superpixel size does not help to overcome inaccurate gaze positions, as the superpixels naturally tend to respect edges and the gaze is in this case usually on the wrong side of the edge. Therefore we did not consider to make our choice of the superpixel size depending on the gaze accuracy. The yellow dots indicate the gaze position.}
	\label{fig:gazeOffSuperpixelSize}
\end{figure}

Certainly, the simple assumption that all the user-indicated positive labels are true positives, is not valid. 
For the instrument data, we gained more true than false positive information from the gaze sequences. 
However, the finer the structures of interest, the less reliably the gaze positions indicate true positives.


% Chapter 4
\chapter{Learning with Unlabeled Data}
\label{chap:learning-with-unlabeled-data}
In this chapter, we will discuss the inherently noisy labels in the negative training set and how to overcome the issues. 
The underlying problem to solve is the following: Assuming we have only true positive samples gained from gaze positions (see Chapter \ref{chap:characterizing-gaze} for a discussion about this assumption), how can we separate positive from negative samples? We formally define the problem and suggest a way to learn from only positive training samples using gradient boosting. Results on synthetic and real data can be found towards the end of the chapter.

\section{Problem Formulation}
The problem of learning a classifier from positive and unlabeled data is to assign labels to the unlabeled dataset \cite{elkan2008learning}. 
It can be considered a semi-supervised learning setup: Instead of having a positive and a negative set of examples, we are given an incomplete set of positives and a set of unlabeled examples. 
The unlabeled data contains positive and negative examples which we want to assign to either the positive or negative class. 
Usually, gradient boosting is used to minimize a loss function $L(y,f(x))$, where $y \in \{-1,1\}^m$ is a vector of labels and $x \in \mathbb{R}^{m\times n}$ is a matrix containing $n$-dimensional features. In our case of the PU-learning problem, however, not all the labels $y_i, i \in \{1,\dots,m\}$ are given. 
Instead we have only an incomplete set of positive labels ($+1$) and the rest is unknown. In order to handle this problem within a gradient boosting framework, we need some pseudo-labels (see e.g.\ \cite{bennett2002exploiting}, \cite{mallapragada2009semiboost}). We define the pseudo-labels to depend on the probability $p_i$ of an unlabeled training sample $x_i$ to be positive:
\begin{equation*}
 y_i = 
    \begin{cases}
	+1, \quad & \text{if } p_i \geq 0.5, \\
	-1, \quad & \text{if } p_i < 0.5.
      \end{cases}
\end{equation*}
We can easily convert the probability $p_i$ to a probability $\tilde p_i$ of having chosen the correct pseudo-label:
\begin{equation*}
 \tilde p_i = 
    \begin{cases}
	p_i, \quad & \text{if } p_i \geq 0.5, \\
	1-p_i, \quad & \text{if } p_i < 0.5.
      \end{cases}
\end{equation*}

The loss function is then defined as
\begin{equation*}
L(y,f(x)) = \underbrace{\sum_{\{i :~ y_i = 1\}} e^{-y_i f(x_i)}}_{P} \quad + \quad \gamma\underbrace{\sum_{\{i:~ y_i \text{unknown}\}} \left( \tilde p_i e^{-y_i f(x_i)} + (1-\tilde p_i) e^{y_i f(x_i)}\right)}_{U}, 
\end{equation*}
where $y_i$ denotes the (pseudo-)label\footnote{The notation does not distinguish between real labels and pseudo-labels. For the part of the sum with unknown labels, $y_i$ denotes the pseudo-labels.} of sample $x_i$, $\tilde p_i$ is the confidence that the pseudo-label $y_i$ is correct, $f(x_i)$ is the predicted score of the classifier and $\gamma$ is a weight for the U-term that we set to be the relative amount of unlabeled samples in the dataset, $\gamma = n_{\text{unlabeled}}/n_{\text{total}}$ for our experiments. 
In a fully supervised case, the U-term is zero and a standard exponential loss function is optimized. 
For unlabeled data samples, the U-term of the loss function heavily penalizes negative margins, if we are very confident about our pseudo-label ($\tilde p_i \approx 1$) and it will penalize positive margins, if the pseudo-label is very unlikely to be correct ($\tilde p_i \approx 0$). 
Note that this extreme case can by definition of $y_i$ and $\tilde p_i$ not occur and is therefore just of explanatory value. 
The minimum of the loss functions varies and gets closer to zero with a decreasing confidence $\tilde p_i$. If we do not know whether or not the pseudo-label is correct or incorrect ($\tilde p_i = 0.5$), then the minimum is at a margin of exactyl zero and what will be penalized are negative as well as positive margins. For such samples, no decision can be made, as this would mean a decision towards one direction based on a randomly chosen pseudo-label (in our case $y_i = 1$, if $\tilde p_i=0.5$). Figure \ref{fig:ourlossfunctionplot} shows the U-term of the loss function for different values of the probability $\tilde p_i$.

\begin{figure}[ht]
  \centering
  \includegraphics[width=\textwidth]{loss_function_different_p.pdf}	
  \caption{U-term of the loss function for different values of $p$. For samples whose label is most likely incorrect ($p \approx 0$), small margins mean correct decisions (i.e.\ different from the label) and they are therefore rewarded whereas large margins are penalized. In the case of $p \approx 1$, it is the opposite.}
  \label{fig:ourlossfunctionplot}
\end{figure}

The derivative of the loss function with respect to the classifier's scores is then given by 

\begin{equation*}
 \frac{\partial L(y_i,f(x_i))}{\partial f(x_i)} = 
    \begin{cases}
	-y_i e^{-y_i f(x_i)}, & \text{if $y_i = 1$}\\
	-\gamma \cdot \left(y_i \tilde p_i e^{-y_i f(x_i)} - y_i (1 - \tilde p_i) e^{y_i f(x_i)} \right), & \text{if $y_i$ unknown.}
      \end{cases}
\end{equation*}

The key of this method is the usage of the probability $p_i$ for each sample $x_i$ to be a positive sample. 
It gives a natural way to tune the algorithm not to concentrate the same way on all wrongly classified samples, but to instead embrace the fact that there is not always a 100\% certainty that we are working with the correct labels.

\section{Synthetic Data}
We conducted some basic experiments on synthetic data. 
The total size of our synthetic training set contained 160 samples, 80 of which were positive and 80 negative. 
The positive training samples were equally generated from normal distributions $\mathcal{N}(\mu_1,\Sigma_1)$ and $\mathcal{N}(\mu_2, \Sigma_2)$ with 
$$\mu_1= \begin{bmatrix}2 \\ 3 \end{bmatrix}, \quad \Sigma_1 = \begin{bmatrix}0.7 & 0.2 \\ 0.2 & 0.5 \end{bmatrix}, \qquad \mu_2 = \begin{bmatrix}4.5 \\ 2 \end{bmatrix}, \quad \Sigma_2 = \begin{bmatrix} 0.2 & 0 \\ 0 & 0.2 \end{bmatrix}$$
and the 80 negative samples were generated from a normal distribution with parameters 
$$\mu_3 = \begin{bmatrix} 2\\1.5\end{bmatrix}, \quad \Sigma_3 = \begin{bmatrix}0.6 & 0.1\\ 0.1 & 0.7\end{bmatrix}$$
as visualized in Figure \ref{fig:synthetic_train_data}. The test set consists of 1000 samples that are identically distributed as the training set.
\begin{figure}[ht]
	\centering
	\begin{subfigure}[h]{0.45\textwidth}
	\includegraphics[width=\textwidth]{synthetic-gaussians-contour.pdf}	
	\end{subfigure}
	~
	\begin{subfigure}[h]{0.45\textwidth}
	\includegraphics[width=\textwidth]{synthetic-gaussians-surf.pdf}	
	\end{subfigure}
	\caption{The used distributions to generate the training and test data used for the synthetic experiments. The positive samples were distributed according to the distributions with the means at the top and on the right side in the left Figure, and the negative samples were distributed according to the contours at the lower left.}
	\label{fig:synthetic-gaussians}
\end{figure}

As a reference (blue curves in Figure \ref{fig:synthetic_results}, we optimized a standard exponential loss using all the labels from the training set with a gradient boosting method with decision tree stumps as weak learners and a shrinkage factor of $0.1.$ 
The other experiments were done with the same algorithm, but different assumptions about the available input labels. 
First, we simulated the case of separating observed from unobserved samples, i.e.\ we tried to separate a few positive samples from all the other samples. 
In the real setting, this corresponds to the naive approach used by Vilari\~no et al.\ \cite{vilarino2007automatic} of separating the image regions that were hit by the user's gaze from all the other regions. 
As expected, there is a considerable performance loss when using this approach (see the red curve in Figure \ref{fig:synthetic_results}). 
Our second experiment (yellow curves in Figure \ref{fig:synthetic_results}) shows the performance achieved with the standard exponential loss using 5 known positives and some pseudo-labels for the other samples. 
The pseudo-labels were assigned according to the probabilities of the known underlying distributions (see Figure \ref{subfig:pu_train}). 
To test the performance of our PU-loss function, we used the same 5 known positives and probabilities. 
Our loss-function outperforms the reference approach using the true training labels as well as the standard exponential loss with the ``correct'' pseudo-labels (that is, the pseudo-labels according to the underlying distributions). 
This can be explained with the fact that our loss function actually takes into account the confidence about the chosen pseudo-labels and adjusts the penalties accordingly. 
Optimizing the double hinge loss and its composite as suggested in \cite{plessis2015convex} (green curve) yielded better results than separating observed from unobserved data points, but could not outperform our loss-function. 

\begin{figure}[ht]
	\centering
	\begin{subfigure}[h]{0.49\textwidth}
	\includegraphics[width=\textwidth]{synthetic_train_data.pdf}	
		\caption{Training data (160 samples generated from the distributions illustrated in Figure \ref{fig:synthetic-gaussians})\newline}
		\label{subfig:ref_train}
	\end{subfigure}
	~
	\begin{subfigure}[h]{0.49\textwidth}
	\includegraphics[width=\textwidth]{synthetic_pu_train_data.pdf}	
		\caption{Input for the PU-loss: 5 known positives (blue), pseudo-labels (red = -1 / green = +1), probability weights $\tilde p_i$ that pseudo-labels are correct (circle radius)}
		\label{subfig:pu_train}
	\end{subfigure}
	\caption{\subref{subfig:ref_train}) Training data for the standard exponential loss. \subref{subfig:pu_train}) Known positives, pseudo-labels and weights used with the standard exponential loss and the PU-loss, respectively.}
	\label{fig:synthetic_train_data}
\end{figure}

\begin{figure}[ht]
	\centering
	\begin{subfigure}[h]{0.49\textwidth}
	\includegraphics[width=\textwidth]{synthetic_results_roc.pdf}	
	\end{subfigure}
	~
	\begin{subfigure}[h]{0.49\textwidth}
	\includegraphics[width=\textwidth]{synthetic_results_pr.pdf}	
	\end{subfigure}
	\caption{Comparisons of results using different assumptions about available labels. The PU-loss outperforms the standard exploss, even when using the theoretically correct pseudo-labels.}
	\label{fig:synthetic_results}
\end{figure}


\section{On the Real Data}
\label{sec:real-data}
\subsection{Features}
Each of our real datasets consist of multiple images that we pre-segmented using the SLIC algorithm (\cite{achanta2010slic}, \cite{vedaldi08vlfeat}) with parameters as indicated in Table \ref{tab:slic-params}. 

\begin{table}[ht]
	\centering
	  \caption{Chosen parameters to pre-segment the input images into SLIC superpixels. The color images from the instrument dataset were converted to CIELAB color space before SLIC was applied.}
	  \label{tab:slic-params}
	\begin{tabular}{ | c  c  c | }
	\hline
	  dataset	& region size & regularizer \\ \hline
	  instrument  	& 35 & 300 \\
	  eye tumor  	& 23 & 0.05 \\ 
	  cochlea	& 22 & 0.02 \\ 
	  airplane 	& 29 & 40 \\ \hline
	\end{tabular}
\end{table}

We decided to use simple descriptors for the superpixels; for the color videos (airplane from Figure \ref{fig:airplaneSLIC} and instrument dataset), we used three-dimensional features
\begin{equation*}
 x = \left( \frac{\bar r}{\bar g + \bar r + 10^{-5}}, \quad \frac{\bar r }{\bar b +\bar r + 10^{-5}}, \quad \frac{\bar g}{ \bar b + \bar g + 10^{-5}}\right),
\end{equation*}
$\bar r, \bar g$ and $\bar b$ being the average of red, green and blue values within the superpixel, respectively. 
For the eye tumor and the cochlea datasets we used the following values to describe a superpixel:
\begin{itemize}
 \item an intensity histogram (10 bins),
 \item average intensity,
 \item variance of intensity,
 \item and the co-occurrence matrix (\cite{haralick1973textural}) of the image when masked to see only the superpixel.
\end{itemize}

\subsection{Probabilities}
Using synthetic data, we could profit from knowing the underlying distribution and directly use this knowledge by feeding the true probabilities to our algorithm. 
However, in reality these probabilities are unknown and we have to come up with a way to estimate them. 
The results presented in this thesis were obtained with experimentally found values; given the observed positive superpixels $x_1,\dots,x_m$ from a gaze sequence, we calculated a distance vector $(d_1,\dots,d_m)$ for every unlabeled sample $\tilde x$ with $d_i = d(\tilde x, x_i), i = 1,\dots,m.$
Here, $d$ is a general distance metric. We used euclidean and cosine metrics as indicated in Table \ref{tab:probabilities}. 
The distance vector was then aggregated to one value $d_{\text{f}}$ by taking the median or minimum value. 
Further, we calculated the spatial (euclidean) distance between every unlabeled superpixel and the gaze position in the corresponding frame 
$$d_\text{s} = d\left(\begin{bmatrix} \text{gaze}_x \\ \text{gaze}_y \end{bmatrix}, \begin{bmatrix} x \\ y \end{bmatrix}\right),$$
where $\begin{bmatrix} x \\ y \end{bmatrix}$ is the median position of all the pixels contained in the unlabeled superpixel.
The probabilities were then calculated as indicated in Table \ref{tab:probabilities}.
There is always a spatial term and a feature term involved. The feature term is based on the distance $d_{\text{f}}$ and therefore includes information about all the positives that were collected using the gaze information. The exact parameters and the usage of different metrics was determined experimentally and there is no guarantee that they are the optimal ones.
\begin{table}[ht]
	\centering
	  \caption{A listing of the formulas used for the different datasets to calculate approximate probability values for the unknown labels to be positive.}
	  \label{tab:probabilities}
	\begin{tabular}{ | c  c  p{7cm} | }
	\hline
	  dataset	& formula 								& $d_{\text{f}}$  \\ \hline
			& & \\
	  instrument  	& $p_{\tilde x} = e^{-d_{\text{f}}/0.15} \cdot e^{-d_{\text{s}}/400}$ 	& $d_{\text{f}} = $ median of euclidean distances  \\ 
	  & & \\
	  eye tumor	& $p_{\tilde x} = e^{-d_{\text{f}}/0.15} \cdot e^{-d_{\text{s}}/400}$ 	& $d_{\text{f}} = $ median of cosine distances 	  \\ 
	  & & \\
	  cochlea 	& $p_{\tilde x} = e^{-d_{\text{f}}/0.15k} \cdot e^{-d_{\text{s}}/40}$ 	& $d_{\text{f}} = $ miminum of cosine distances; $k = \max(d_\text{f})$ is a normalization constant used because the minimum cosine distances are small numbers\\ \hline
	\end{tabular}
\end{table}

\subsection{Results}
\subsubsection*{Labels inferred from ground truth}
Using manually labeled ground truth data for the three datasets instrument, eye tumor and cochlea, we explored how well the standard exponential loss and our suggested PU-loss perform with different assumptions about available labels. 
Assuming that we have for a certain amount of frames / depth slices the correct superpixel labels\footnote{$y_i = 1$, if more than 50\% of the pixels contained in a superpixel are positive in the ground truth}, we optimized a standard exponential loss with a gradient boosting algorithm using decision tree stumps as weak learners. 
This serves as a reference for our experiments.
We evaluate how well the classifier generalizes to the remaining frames on a pixelwise basis; that is, we classify whole superpixels, but at the end we are interested in the pixelwise segmentation result and compute Receiver Operator Characteristics (ROC) and Precision-Recall values on a pixelwise basis. 
This strategy allows us to compare the results to the real pixelwise ground truth instead of a constructed ground truth like e.g.\ declaring a superpixel as positive, if it contains more than 30\% or 50\% positive ground truth pixels.
For the eye tumor and the cochlea data there are only a few slices containing true positive information, and usually the user presses and holds the key during most of these frames. 
Therefore, evaluating only the generalization to the remaining frames is not informative and, unlike in the instrument dataset, will not tell us enough about the potential of separating the interesting structure from the rest. 
Instead, in these datasets the evaluation has been done for a the whole volume, including already seen superpixels during training. 
When we are later using only gaze observations, we will not exclude the already seen superpixels either. As our ultimate goal is to achieve a full segmentation we have to evaluate every superpixel of every frame.

In our setting, the best we can hope for is one true positive superpixel in every frame. 
To simulate this case with the ground truth data, we choose one true positive superpixel at random from some frames\footnote{The frames were chosen according to be the ones where the user pressed the key during one of the recorded gaze sequences.}, and set the others to negative / unlabeled. We can see in Figure \ref{fig:one-random-tp-per-frame} that our PU-loss function brings a small performance gain compared to the standard exponential loss (red line) with respect to the measured AUC for all the datasets. 
The standard classifier tries to separate observed from non-observed superpixels. In all our datasets, this is not the correct behaviour and it means that the learned classifier holds a bias (this has been discussed e.g.\ by Du Plessis et al.\ (\cite{plessis2014PUanalysis}) and Figure \ref{fig:bias-in-heatmaps} shows that with the standard exponential loss, we have to find a threshold below zero to get any positive labels in the output. 

\begin{figure}[ht]
	\centering
	\begin{subfigure}[h]{0.45\textwidth}
	\includegraphics[width=\textwidth]{d2-one_random_tp_per_frame-roc.pdf}	
		\caption*{instrument}
	\end{subfigure}
	~
	\begin{subfigure}[h]{0.45\textwidth}
	\includegraphics[width=\textwidth]{d2-one_random_tp_per_frame-pr.pdf}	
		\caption*{instrument}
	\end{subfigure}
	
	\vspace{3mm}
	\begin{subfigure}[h]{0.45\textwidth}
	\includegraphics[width=\textwidth]{d7-one_random_tp_per_frame-roc.pdf}	
		\caption*{eye tumor}
	\end{subfigure}
	~
	\begin{subfigure}[h]{0.45\textwidth}
	\includegraphics[width=\textwidth]{d7-one_random_tp_per_frame-pr.pdf}	
		\caption*{eye tumor (tested frames: [12:80])}
	\end{subfigure}	
	
	\vspace{3mm}
	\begin{subfigure}[h]{0.45\textwidth}
	\includegraphics[width=\textwidth]{d8-one_random_tp_per_frame-roc.pdf}	
		\caption*{cochlea}
	\end{subfigure}
	~
	\begin{subfigure}[h]{0.45\textwidth}
	\includegraphics[width=\textwidth]{d8-one_random_tp_per_frame-pr.pdf}	
		\caption*{cochlea (tested frames: [70:250])}
	\end{subfigure}		
	\caption{Randomly chosen true positive for each frame that was observed (i.e.\ where the user pressed the key) in one of the gaze observations.}
	\label{fig:one-random-tp-per-frame}
\end{figure}


\begin{figure}[ht]
	\centering
 	\begin{subfigure}[h]{0.48\textwidth}
	  \includegraphics[width=\textwidth]{d2-frame_00455-input}
	  \caption*{input image \\ (frame 455 of dataset ``instrument'')}
	\end{subfigure}
	~
	\begin{subfigure}[h]{0.48\textwidth}
	  \includegraphics[width=\textwidth]{d2-frame_00455-groundtruth}
	  \caption*{ground truth \\ \quad}
	\end{subfigure}
	
	\vspace{3mm}
	\begin{subfigure}[h]{0.48\textwidth}
	  \includegraphics[width=\textwidth]{d2-frame_00455-reference-heatmap.eps}
	  \caption*{reference: gradient boosting with all the correct positive and negative labels inferred from ground truth}	
	\end{subfigure}
	~
	\begin{subfigure}[h]{0.48\textwidth}
	  \includegraphics[width=\textwidth]{d2-frame_00455-gradboost-heatmap.eps}
	  \caption*{exponential loss \newline}
	\end{subfigure}	
	~
	\begin{subfigure}[h]{0.48\textwidth}
	  \includegraphics[width=\textwidth]{d2-frame_00455-pugradboost-heatmap.eps}
		  \caption*{PU-loss \\ }
		  % 500 iterations, shrinkage = 0.1, stumps
	\end{subfigure}		
	\caption{Resulting scores for one frame using as input random true positives inferred from the ground truth. Whereas both, the standard exponential loss and our PU-loss yield higher scores for regions containing the object, our approach leads to the threshold separating positives from negatives being closer to 0.}
	\label{fig:bias-in-heatmaps}
\end{figure}

\subsubsection*{Labels inferred from gaze observations}
The difference between the above case to our actual situation lies mainly in the distribution of the positive labels -- they are not, as assumed in the previous section, randomly taken from all the positives of the observed frames, but instead they might often over-represent certain positives and under-represent others due to e.g.\ the fact that it is easier to fixate an edge than a smooth region in an image. 
In this setting it becomes more important that not all the unlabeled samples are considered negatives. 
The red curves in Figure \ref{fig:results-curves} show a rather low performance for this approach in each dataset. 
Our loss function on the otherhand enabled us to condsider other superpixels likely to be positive, even if they were never directly observed by the user. 
The performance plots of our method (yellow curves in Figure \ref{fig:results-curves}) show that we outperform the previously presented idea of Vilari\~no et al.\ to consider only one true positive per frame.

\begin{figure}[ht]
	\centering
	\begin{subfigure}[h]{0.45\textwidth}
	\includegraphics[width=\textwidth]{d2-gaze2-results-roc.pdf}	
		\caption*{instrument}
	\end{subfigure}
	~
	\begin{subfigure}[h]{0.45\textwidth}
	\includegraphics[width=\textwidth]{d2-gaze2-results-pr.pdf}	
		\caption*{instrument}
	\end{subfigure}
	
	\vspace{3mm}
	\begin{subfigure}[h]{0.45\textwidth}
	\includegraphics[width=\textwidth]{d7-gaze4-results-roc.pdf}	
		\caption*{eye tumor}
	\end{subfigure}
	~
	\begin{subfigure}[h]{0.45\textwidth}
	\includegraphics[width=\textwidth]{d7-gaze4-results-pr.pdf}	
		\caption*{eye tumor (tested frames: [12:80])}
	\end{subfigure}	
	
	\vspace{3mm}
	\begin{subfigure}[h]{0.45\textwidth}
	\includegraphics[width=\textwidth]{d8-gaze2-results-roc.pdf}	
		\caption*{cochlea}
	\end{subfigure}
	~
	\begin{subfigure}[h]{0.45\textwidth}
	\includegraphics[width=\textwidth]{d8-gaze2-results-pr.pdf}	
		\caption*{cochlea (tested frames: [70:250])}
	\end{subfigure}		
	\caption{Results using only actual gaze observations.}
	\label{fig:results-curves}
\end{figure}

Some resulting scores are shown for each dataset in Figures \ref{fig:results-d2-gaze2}, \ref{fig:results-d7-gaze4} and \ref{fig:results-d8-gaze2}. 
The heat maps in Figure \ref{fig:results-d2-gaze2} show that our approach gives e.g.\ higher scores to the tip of the instrument (best visible in frame 738). 
In frame 29, both approaches fail to recognize one part of the tip of the instrument, most likely due to the pre-segmentation that does not respect this particular edge in the image and therefore superpixel contains more negative (background) than positive (instrument) information. Even though the shaft of the instrument did not get positive scores in the optimized PU-loss, it is separated from the surrounding regions, whereas the standard exponential yields non-uniform and rather scores for those superpixels (best visible in frame 29).
In the eye tumor dataset (Figure \ref{fig:results-d7-gaze4}), we see that more superpixels get higher scores whereas the standard exponential loss leads mostly to only one positive per frame (the one that was hit by the gaze) and the others have rather low scores. 
Our approach yields as well the same positives, but shows more uncertainty about the (wrong) negative predictions at the locations of the other true positives. 
The same holds for the example of the cochlea in Figure \ref{fig:results-d8-gaze2}. 
However, we can here as well see that our approach is heavily dependent on the gaze-positions. As elaborated in chapter \ref{chap:characterizing-gaze}, the gaze sequences for the latter two datasets contain less true positive information than for the first dataset. However, in our problem formulation, we assumed that the given extracted superpixels are true positives. For the unlabeled superpixels, the pseudo-labels were estimated based on very noisy information from the extracted positives of the whole sequence. Therefore the final segmentation results for those cases are not satisfying at this stage.

\begin{figure}[ht]
	\centering
	\begin{subfigure}[h]{0.32\textwidth}
	\includegraphics[width=\textwidth]{d2-frame_00121-input-segmented}	
		\caption*{frame 121}
	\end{subfigure}
	~
	\begin{subfigure}[h]{0.32\textwidth}
	\includegraphics[width=\textwidth]{d2-frame_00738-input-segmented}	
		\caption*{frame 738}
	\end{subfigure}
	~
	\begin{subfigure}[h]{0.32\textwidth}
	\includegraphics[width=\textwidth]{d2-frame_00029-input-segmented}	
		\caption*{frame 29}
	\end{subfigure}
	
	\vspace{3mm}
	\begin{subfigure}[h]{0.32\textwidth}
	\includegraphics[width=\textwidth]{d2-frame_00121-groundtruth}	
	\end{subfigure}
	~
	\begin{subfigure}[h]{0.32\textwidth}
	\includegraphics[width=\textwidth]{d2-frame_00738-groundtruth}	
	\end{subfigure}
	~
	\begin{subfigure}[h]{0.32\textwidth}
	\includegraphics[width=\textwidth]{d2-frame_00029-groundtruth}	
	\end{subfigure}
	
	\vspace{3mm}
	\begin{subfigure}[h]{0.32\textwidth}
	\includegraphics[width=\textwidth]{d2-frame_00121-gaze2-gradboost-heatmap.eps}	
	\end{subfigure}
	~
	\begin{subfigure}[h]{0.32\textwidth}
	\includegraphics[width=\textwidth]{d2-frame_00738-gaze2-gradboost-heatmap.eps}
	\end{subfigure}
	~	
	\begin{subfigure}[h]{0.32\textwidth}
	\includegraphics[width=\textwidth]{d2-frame_00029-gaze2-gradboost-heatmap.eps}	
	\end{subfigure}
	
	\vspace{3mm}
	\begin{subfigure}[h]{0.32\textwidth}
	\includegraphics[width=\textwidth]{d2-frame_00121-gaze2-pugradboost-heatmap.eps}	
	\end{subfigure}
	~
	\begin{subfigure}[h]{0.32\textwidth}
	\includegraphics[width=\textwidth]{d2-frame_00738-gaze2-pugradboost-heatmap.eps}	
	\end{subfigure}
	~	
	\begin{subfigure}[h]{0.32\textwidth}
	\includegraphics[width=\textwidth]{d2-frame_00029-gaze2-pugradboost-heatmap.eps}	
	\end{subfigure}	
	
	\caption{Heat maps showing resulting scores using only actual gaze observations. Top row: Input superpixels. Second row: Ground truth. Third row: resulting scores using standard exponential loss. Bottom row: resulting scores using the PU-loss.}
	\label{fig:results-d2-gaze2}
\end{figure}

\begin{figure}[ht]
	\centering
	\begin{subfigure}[h]{0.32\textwidth}
	\includegraphics[width=\textwidth]{d7-frame_00043-input-segmented}	
		\caption*{frame 43}
	\end{subfigure}
	~
	\begin{subfigure}[h]{0.32\textwidth}
	\includegraphics[width=\textwidth]{d7-frame_00046-input-segmented}	
		\caption*{frame 46}
	\end{subfigure}
	~
	\begin{subfigure}[h]{0.32\textwidth}
	\includegraphics[width=\textwidth]{d7-frame_00048-input-segmented}	
		\caption*{frame 48}
	\end{subfigure}
	
	\vspace{3mm}
	\begin{subfigure}[h]{0.32\textwidth}
	\includegraphics[width=\textwidth]{d7-frame_00043-groundtruth}	
	\end{subfigure}
	~
	\begin{subfigure}[h]{0.32\textwidth}
	\includegraphics[width=\textwidth]{d7-frame_00046-groundtruth}	
	\end{subfigure}
	~
	\begin{subfigure}[h]{0.32\textwidth}
	\includegraphics[width=\textwidth]{d7-frame_00048-groundtruth}	
	\end{subfigure}
	
	\vspace{3mm}
	\begin{subfigure}[h]{0.32\textwidth}
	\includegraphics[width=\textwidth]{d7-frame_00043-gaze4-gradboost-heatmap.eps}	
	\end{subfigure}
	~
	\begin{subfigure}[h]{0.32\textwidth}
	\includegraphics[width=\textwidth]{d7-frame_00046-gaze4-gradboost-heatmap.eps}
	\end{subfigure}
	~	
	\begin{subfigure}[h]{0.32\textwidth}
	\includegraphics[width=\textwidth]{d7-frame_00048-gaze4-gradboost-heatmap.eps}	
	\end{subfigure}
	
	\vspace{3mm}
	\begin{subfigure}[h]{0.32\textwidth}
	\includegraphics[width=\textwidth]{d7-frame_00043-gaze4-pugradboost-heatmap.eps}	
	\end{subfigure}
	~
	\begin{subfigure}[h]{0.32\textwidth}
	\includegraphics[width=\textwidth]{d7-frame_00046-gaze4-pugradboost-heatmap.eps}	
	\end{subfigure}
	~	
	\begin{subfigure}[h]{0.32\textwidth}
	\includegraphics[width=\textwidth]{d7-frame_00048-gaze4-pugradboost-heatmap.eps}	
	\end{subfigure}	
	
	\caption{Heat maps showing resulting scores using only actual gaze observations. Top row: Input superpixels. Second row: Ground truth. Third row: resulting scores using standard exponential loss. Bottom row: resulting scores using the PU-loss.}
	\label{fig:results-d7-gaze4}
\end{figure}

\begin{figure}[ht]
	\centering
	\begin{subfigure}[h]{0.32\textwidth}
	\includegraphics[width=\textwidth]{d8-frame_00170-input-segmented}	
		\caption*{frame 170}
	\end{subfigure}
	~
	\begin{subfigure}[h]{0.32\textwidth}
	\includegraphics[width=\textwidth]{d8-frame_00193-input-segmented}	
		\caption*{frame 193}
	\end{subfigure}
	~
	\begin{subfigure}[h]{0.32\textwidth}
	\includegraphics[width=\textwidth]{d8-frame_00197-input-segmented}	
		\caption*{frame 197}
	\end{subfigure}

	\vspace{3mm}
	\begin{subfigure}[h]{0.32\textwidth}
	\includegraphics[width=\textwidth]{d8-frame_00170-groundtruth}	
	\end{subfigure}
	~
	\begin{subfigure}[h]{0.32\textwidth}
	\includegraphics[width=\textwidth]{d8-frame_00193-groundtruth}	
	\end{subfigure}
	~
	\begin{subfigure}[h]{0.32\textwidth}
	\includegraphics[width=\textwidth]{d8-frame_00197-groundtruth}	
	\end{subfigure}		
	
	\vspace{3mm}
	\begin{subfigure}[h]{0.32\textwidth}
	\includegraphics[width=\textwidth]{d8-frame_00170-gaze2-gradboost-heatmap.eps}	
	\end{subfigure}
	~
	\begin{subfigure}[h]{0.32\textwidth}
	\includegraphics[width=\textwidth]{d8-frame_00193-gaze2-gradboost-heatmap.eps}
	\end{subfigure}
	~	
	\begin{subfigure}[h]{0.32\textwidth}
	\includegraphics[width=\textwidth]{d8-frame_00197-gaze2-gradboost-heatmap.eps}	
	\end{subfigure}
	
	\vspace{3mm}
	\begin{subfigure}[h]{0.32\textwidth}
	\includegraphics[width=\textwidth]{d8-frame_00170-gaze2-pugradboost-heatmap.eps}	
	\end{subfigure}
	~
	\begin{subfigure}[h]{0.32\textwidth}
	\includegraphics[width=\textwidth]{d8-frame_00193-gaze2-pugradboost-heatmap.eps}	
	\end{subfigure}
	~	
	\begin{subfigure}[h]{0.32\textwidth}
	\includegraphics[width=\textwidth]{d8-frame_00197-gaze2-pugradboost-heatmap.eps}	
	\end{subfigure}	
	
	\caption{Heat maps showing resulting scores using only actual gaze observations. Top row: Input superpixels. Second row: Ground truth. Third row: resulting scores using standard exponential loss. Bottom row: resulting scores using the PU-loss.}
	\label{fig:results-d8-gaze2}
\end{figure}




%mainly because it means that a superpixel that has been labeled positive will have a multitude of negatively labeled ``opponents'' that look almost the same. It can be clearly seen that positions that have been observed only for a short time, even though they belong to the positive set, are not well separated from the negative set because very similar parts will end up in the negative set.





% Chapter 5
\chapter{Conclusion}
\label{chap:conclusion}
\section{Conclusion}
Gaining positive labels using eye-tracking is a realistic idea and it can be used to segment images by interpreting the problem as a semi-supervised learning problem with only positive and unlabeled training samples. The simple assumption that all the user-indicated positive labels are true positives, is not valid. For the instrument data, we gained more true than false positive information from the gaze sequences. However, the finer the structures of interest are, the less reliably the gaze positions indicate true positives, as became evident in Chapter \ref{chap:characterizing-gaze}. Further, we realized that not only the absolut size, but also the rate of change influences the ability of the user to fixate the object in the video. This observation suggests that the reliability can be improved, for example by giving the user control about the playback mode of the video sequences.

We demonstrated that the previous approach of Vilari\~no et al.\ \cite{vilarino2007automatic} is not directly applicable to arbitrary data and reformulated the underlying problem to overcome the restrictions. 
The suggested loss function can be used within a standard gradient boosting framework. 
A drawback of our approach is that providing the correct probabilities to define the loss function is not a trivial task. Yet, these probability values are the key to our method and allow us to include arbitrary information about the whole dataset into our standard loss function; we used distances metrics in spatial and feature domain combining information of the current frame with the whole video sequences. For synthetic as well as for real data, our approach improves the classification / segmentation results.

\section{Future work}

\subsection{Feature Learning}
The features we used for this work were basic hand-engineered ones. 
A strategy that could lead to better descriptors might be to learn features in an unsupervised way. 
For example \cite{ng2011sparse} describes nicely how to use sparse autoencoders to learn features from unlabeled data. This strategy was shown to be helpful in training deep neural networks without getting stuck in local minima \cite{bengio2007greedy}.

\subsection{Use Consistency}
The data we are dealing with in this application are usually smoothly changing over time.
While we did consider information from the whole sequences / volumes by including it in our probabilities (Section \ref{sec:real-data}), smoothness constraints could possibly be used to identify gaze positions that are unlikely to indicate true positive data. Also, label smoothing as a post-processing step (see for example \cite{zhou2004learning}) could be considered to enforce smoothness.
Using supervoxels instead of superpixels could further ensure consistency along the third dimension. Supervoxels have successfully been used for video segmentation for example by Corso et al.\ \cite{CoShDuTMI2008} and Grundmann et al.\ \cite{grundmann2010efficient}.

\subsection{Reliability of Gaze Sequences}
Building an interface that allows the user to control the playback mode of the video while the gaze is observed could significantly improve the reliability of the gaze sequences. 
Further, the potential of the approach to become a passive instead of an active application of eye-tracking should be investigated. 
A passive setup would simplify the handling of such a system for clinicians and might bring us towards gathering annotations using the specialist's knowledge, without him being required to actively provide us with accurate gaze fixations.

%END Doc
%-------------------------------------------------------

\bibliography{thesis}
\bibliographystyle{plain}

\end{document}
